\documentclass{report}

\input{~/dev/latex/template/preamble.tex}
\input{~/dev/latex/template/macros.tex}

\title{\Huge{Calculus 1: Chapter 3}}
\author{\huge{Nathan Warner}}
\date{\huge{Feb 12, 2023}}

\pgfpagesdeclarelayout{boxed}
{
  \edef\pgfpageoptionborder{0pt}
}
{
  \pgfpagesphysicalpageoptions
  {%
    logical pages=1,%
  }
  \pgfpageslogicalpageoptions{1}
  {
    border code=\pgfsetlinewidth{1.5pt}\pgfstroke,
    border shrink=\pgfpageoptionborder,
    resized width=.95\pgfphysicalwidth,
    resized height=.95\pgfphysicalheight,
    center=\pgfpoint{.5\pgfphysicalwidth}{.5\pgfphysicalheight}
  }
}

\pgfpagesuselayout{boxed}

\begin{document}
    \maketitle
    \begin{center}
        \begin{Huge}
            \textbf{Chapter 3}
        \end{Huge}
    \end{center}
    \line(1,0){490}
    \bigbreak \noindent \bigbreak \noindent  
    \begin{Huge}
        \noindent \textbf{Contents}
    \end{Huge}
    \bigbreak \noindent \bigbreak \noindent 
    \begin{Large}
        \textbf{3.1: Differential Rule}
        \bigbreak \noindent \bigbreak \noindent 
        \textbf{3.2: The Product and Quotient Rule}
        \bigbreak \noindent \bigbreak \noindent 
        \textbf{3.3: Derivatives of Trigonometric Functions}
        \bigbreak \noindent \bigbreak \noindent 
        \textbf{3.4.1: The Chain Rule}
        \bigbreak \noindent \bigbreak \noindent 
        \textbf{3.4.2: Differentiation Examples using the Product, Quotient, and Chain Rules}
        \bigbreak \noindent \bigbreak \noindent 
        \textbf{3.5: Implicit Differentiation and 3.6 (Part 1) Derivatives of Inverse Trigonometric Functions}
        \bigbreak \noindent \bigbreak \noindent 
        \textbf{3.6: (Part 2) Derivatives of Logarithmic Functions}
        \bigbreak \noindent \bigbreak \noindent 
        \textbf{3.7: Rates of Change in the Natural and Social Sciences}
        \bigbreak \noindent \bigbreak \noindent 
        \textbf{3.8: Exponential Growth and Decay -- Newton's Law of Cooling}
        \bigbreak \noindent \bigbreak \noindent 
        \textbf{3.9: Related Rates}
        \bigbreak \noindent \bigbreak \noindent 
        \textbf{3.10: Linear Approximations and Differentials}
        \bigbreak \noindent \bigbreak \noindent 
        \textbf{3.11: Hyperbolic Functions}
    \end{Large}

    \pagebreak \bigbreak \noindent
    \begin{Large}
        \begin{mdframed}
            \begin{center}
                \textbf{3.1}
            \end{center}
        \end{mdframed}
    \end{Large}
    \begin{Large}
        \begin{center}
            \textbf{Differential Rule:}
        \end{center}
    \end{Large}
    \line(1,0){490}
    
    \bigbreak \noindent 
    \begin{mdframed}
        \textbf{\textit{Diffential Fomulas:}}
        \begin{itemize}
            \item $ \frac{d}{dx}(c) = 0$
            \item $ \frac{d}{dx}(x) = 1$
            \item $ \frac{d}{dx}(x^n) = n \cdot x^{n-1} \rightarrow$ \text{\textbf{\textit{Power Rule}}}
            \item $ \frac{d}{dx}[c \cdot f(x)] = c \cdot \frac{d}{dx}[f(x)]$
            \item $ \frac{d}{dx}[f(x) \pm g(x)] = \frac{d}{dx}f(x)\pm \frac{d}{dx}g(x)$
        \end{itemize}
    \end{mdframed}
    \bigbreak \noindent \bigbreak \noindent 
    \ex{}{Differentiate the following functions:} 
    \bigbreak \noindent 
    \begin{mdframed}
       \textbf{1.)} $f(t) = \frac{1}{2}t^6 - 3t^4 +1$ 
    \end{mdframed}
  
   \bigbreak \noindent  \bigbreak \noindent 
   For the first term, we will use the \textbf{\textit{Third and Fourth}} Rule:
   \begin{align*}
      \frac{1}{2} \cdot 6t^{t-1}
   .\end{align*}
   \bigbreak \noindent 
   For the second term, \textit{$-3t^4$}, We will use the \textbf{\textit{Third and Fifth}} Rule:
   \begin{align*}
      -3 \cdot 4t^{4-1} 
   .\end{align*}
   \bigbreak \noindent 
   The last term is a constant, so according to the first rule, the Derivative of a constant is \textbf{\textit{Zero}}:

  \begin{center}
    \textit{So our full equation is:}
  \end{center}
  \begin{align*}
    f\prime(x) = \frac{1}{2} \cdot 6t^{6-1} - 3 \cdot 4t^{4-1} + 0 \\ 
    = 3t^5-12t^3 
  .\end{align*}
  
  \bigbreak \noindent \bigbreak \noindent 
  \begin{mdframed}
    \textbf{2.)} $h(x) = (x-2)(2x+3)$ 
  \end{mdframed}
  First we need to distribute out the terms:
  \begin{align*}
    h(x) = 2x^2 + 3x -4x -6 \\ 
    = 2x^2 -x-6
  .\end{align*}
  \bigbreak \noindent 
  \begin{center}
    Now this is the function we want to diferentiate.
  \end{center}
  \bigbreak \noindent 
  \textit{So $\rightarrow$}
  \begin{align*}
    h\prime(x) = 2 \cdot 2x^{2-1} - 1 - 0  \\ 
    h\prime(x) = 4x -1
  .\end{align*}
  
  \begin{mdframed}
    \textbf{3.)} $y = \frac{x^2 - 2 \sqrt{x}}{x}$
  \end{mdframed}
  \bigbreak \noindent 
  \textit{So:}
  \begin{align*}
      y = \frac{x^2 - 2 x^{\frac{1}{2}}}{x}
  .\end{align*}
  \bigbreak \noindent 
  \begin{center}
    \textit{Since the denominator only has \textbf{one term}, we can split the equation like:}
  \end{center}
  \begin{align*}
    y = \frac{x^2}{x} - \frac{2x^{ \frac{1}{2}}}{x} \\ 
    y = x - 2x^{- \frac{1}{2}}
  .\end{align*}
  \bigbreak \noindent 
  \textit{Now:}
  \begin{align*}
    \frac{dy}{dx} = 1 - 2 \cdot (-\frac{1}{2})x^{- \frac{1}{2} - 1} \\ 
    \frac{dy}{dx} = 1 + x^{- \frac{3}{2}}
  .\end{align*}
  \bigbreak \noindent 
  \textit{And we can even rewrite it as:}
  \begin{align*}
    \frac{dy}{dx} = 1 + \frac{1}{x^{ \frac{3}{2}}}  
  .\end{align*}
  
  \bigbreak \noindent \bigbreak \noindent  
  \begin{mdframed}
    \textbf{4.)} $ V = ( \sqrt{x} + \frac{1}{ \sqrt[3]{x}})^2$
  \end{mdframed}
  \bigbreak \noindent 
  \textit{So:}
  \begin{align*}
    V = (x^{ \frac{1}{2}} + x^{ \frac{-1}{3}})^2 \\ 
    = (x^{ \frac{1}{2}})^2  + 2(x^{ \frac{1}{2}})(x^{- \frac{1}{3}}) + (x^{- \frac{1}{3}})^2 \\
    = x + 2x^{ \frac{1}{6}} + x^{ - \frac{2}{3}}
  .\end{align*}
  \bigbreak \noindent 
  \textit{Now we find the Derivative:}
  \begin{align*}
    V\prime = 1 + 2 \cdot \frac{1}{6} x^{ \frac{1}{6} - 1} + ( \frac{-2}{3})x^{- \frac{2}{3} -1} \\ 
    v\prime = 1 +  \frac{1}{3}x^{- \frac{5}{6}} - \frac{2}{3}x^{- \frac{5}{3}} \\ 
    v\prime = 1 + \frac{1}{3x^{ \frac{5}{6}}} - \frac{2}{3x^{ \frac{5}{3}}}
  .\end{align*}

  \pagebreak \bigbreak \noindent
  \begin{large}
    \textbf{Exponential Functions:}
  \end{large}

  \bigbreak \noindent \bigbreak \noindent 
  \textbf{Recall:} $(1+ \frac{1}{n})^n \rightarrow e \approx 2.71828... as n \rightarrow \infty $

  \bigbreak \noindent \bigbreak \noindent  
  \dfn{Definiton of e:}{ $\lim\limits_{h \to 0}{ \frac{e^h -1}{h}} = 1$}
  \bigbreak \noindent 
  \nt{\textit{We'll use the above definiton to derive $ \frac{d}{dx}(e^x)$}}

  \bigbreak \noindent 
  $\rightarrow$ Let $f(x) = e^x$
  \begin{align*}
    f\prime(x) = \lim\limits_{h \to 0}{ \frac{f(x+h) - f(x)}{h}}
  .\end{align*}
  \textit{So:}
  \begin{align*}
    \lim\limits_{h \to 0}{ \frac{e^{x+h} - e^x}{h}} \\ 
    = \lim\limits_{h \to 0}{ \frac{e^x \cdot x^h - e^x} {h}} \\ 
    = \lim\limits_{h \to 0}{ \frac{e^x \cdot (e^h-1)}{h}}
  .\end{align*}
  \bigbreak \noindent 
  \textit{This function is dependent on \textbf{h}, but $e^x$ is not dependent on h, so we can pull it outside and rewrite as:}
  \begin{align*}
    e^x \cdot \lim\limits_{h \to 0}{ \frac{e^h-1}{h}}
  .\end{align*}
  \textit{According to our definiton above, we can see that the right portion of this equation \textbf{\textit{Equals 1}}, Therefor we are just left with:}
  \begin{align*}
    e^x
  .\end{align*}
  \bigbreak \noindent 
  \textit{Therefore:}
  \begin{align*}
    \frac{d}{dx}(e^x) = e^x
  .\end{align*}

  \bigbreak \noindent 
  \begin{mdframed}
    \textbf{Example:} Find $f\prime(x)$ and $f\prime\prime(x)$ of $f(x) = e^x - x^3$ 
  \end{mdframed}
  \bigbreak \noindent 
  \begin{align*}
    f\prime(x) = e^x - 3x^2
  .\end{align*}
  \bigbreak \noindent 
  \begin{align*}
    f\prime\prime(x) = e^x - 6x
  .\end{align*}

  \pagebreak \bigbreak \noindent
  \begin{large}
    \textbf{Normal Line:}
  \end{large} 
  \bigbreak \noindent 
  The normal line is perpendicular to the tangent line at the point of tangency.
  \begin{align*}
    m_{tangent} \cdot m_{normal} = -1
  .\end{align*}
  \bigbreak \noindent 
  \nt{This definition means that the slopes are \textbf{\textit{Opposite Recipricals}}}
  \bigbreak \noindent 
  \begin{mdframed}
    \textbf{Example:} find equations of the tangent line and the normal line to the curve $y=x^4 + 8e^x$ at the 
    point (0,8).
  \end{mdframed}
  \bigbreak \noindent 
  \textit{So we find the derivative:}
  \begin{align*}
    y\prime = 4x^3 + 8e^x
  .\end{align*}
  \bigbreak \noindent 
  \textit{Then we find $m_{tan}$:}
  \begin{align*}
    m_{tan} = 4 \cdot 0^3 + 8e^0 \\ 
    = 0 + 8 \cdot 1 \\ 
    = 8
  .\end{align*}
  \bigbreak \noindent 
  \textit{Then we find the slope of the normal line, so we take the Reciprical of $m_{tan}$, so we \textbf{\textit{flip it and change the sign}}:}
  \begin{align*}
    m_{normal} = - \frac{1}{8} 
  .\end{align*}
  \bigbreak \noindent 
  We can check our answer using the definiton:
  \begin{align*}
    8(- \frac{1}{8}) = -1
  .\end{align*}
  \bigbreak \noindent 
  \textit{Now we find the equations of the lines:}
  \begin{center}
    \textbf{Tangent Line:}
  \end{center}
  \begin{align*}
    y - 8 = 8(x - 0) \\ 
    y - 8 = 8x \\ 
    y = 8x + 8
  .\end{align*}
  \bigbreak \noindent 
  \begin{center}
    \textbf{Normal Line:}
  \end{center}
  \begin{align*}
    y - 8 = - \frac{1}{8}(x-0) \\ 
    y - 8 = - \frac{1}{8}x \\ 
    y = - \frac{1}{8}x + 8 
  .\end{align*}

  \pagebreak \bigbreak \noindent
  \begin{mdframed}
    \textbf{Example:} The equation of motion of a particle is $s = t^3 -12t$
  \end{mdframed}
  \bigbreak \noindent 
  \textbf{a.)} Find $v(t) = s\prime(t)$ - \textit{Velocity}
  \bigbreak \noindent 
  \textit{So:}
  \begin{align*}
    s\prime(t) = 3t^2 -12
  .\end{align*}
  \bigbreak \noindent 
  \textbf{B.)} Find $a(t) = s\prime\prime(t)$ - \textit{Acceleration}
  \bigbreak \noindent 
  \textit{So:}
  \begin{align*}
    s\prime\prime(t) = 6t 
  .\end{align*}
  \bigbreak \noindent 
  \textbf{c.)} Find the acceleration after 9 seconds
  \bigbreak \noindent 
  \textit{So:}
  \begin{align*}
   a(9) = 6 \cdot 9 \\
  = 54 m \diagdown s^2
  .\end{align*}
  \bigbreak \noindent 
  \textbf{d.)} Find the acceleration when the velocity is 0.
  \bigbreak \noindent 
 \textit{So:} 
 \begin{center}
   Set v(t) = 0 
 \end{center}
 \begin{align*}
   3t^2 -12 = 0 \\ 
   3t^2 = 12 \\ 
   t^2 = 4 \\ 
   t = \pm 2 \rightarrow 2\ \text{Typically we like t to be positive}
 .\end{align*}
 \bigbreak \noindent 
  \textit{Now:}
  \begin{align*}
    a(2) = 6 \cdot 2 \\ 
    = 12 m \diagdown s^2
  .\end{align*}

  \pagebreak \bigbreak \noindent
  \begin{Large}
      \begin{mdframed}
          \begin{center}
              \textbf{3.2}
          \end{center}
      \end{mdframed}
  \end{Large}
  \begin{Large}
      \begin{center}
          \textbf{The Product and Quotient Rules}
      \end{center}
  \end{Large}
  \line(1,0){490}
  
  \bigbreak \noindent \bigbreak \noindent \bigbreak \noindent 
  \begin{large}
    \textbf{Product Rule:}
  \end{large}
  \begin{mdframed}
    \begin{align*}
      \frac{d}{dx}[f(x) \cdot g(x)] = f(x) \frac{d}{dx}[g(x)] + g(x) \frac{d}{dx}[f(x)]
    .\end{align*}
    \begin{center}
      \textbf{\textit{Or:}}
    \end{center}
    \begin{align*}
      (f\cdot g)\prime = f \cdot g\prime + g \cdot f\prime
    .\end{align*}
  \end{mdframed}
  \bigbreak \noindent 
  \begin{large}
    \textbf{Quotient Rule:}
  \end{large}
  \begin{mdframed}
    \begin{align*}
      \frac{d}{dx}\bigg[ \frac{f(x)}{g(x)}\bigg] = \frac{g(x) \frac{d}{dx}[f(x)] - f(x) \frac{d}{dx}[g(x)]}{[g(x)]^2}
    .\end{align*}
    \begin{center}
      \textbf{Or:}
    \end{center}
    \begin{align*}
      \bigg(\frac{f}{g}\bigg)^{\prime} = \frac{g \cdot f ^{\prime} - f \cdot g ^{\prime}}{g^2}
    .\end{align*}
  \end{mdframed}

  \bigbreak \noindent \bigbreak \noindent 
  \begin{mdframed}
    \textbf{Example:} Differentiate the following Function: \textbf{(Quotient Rule)}
  \end{mdframed}
  \bigbreak \noindent 
  \textbf{1.)} $y = \frac{e^x}{1+x}$

  \bigbreak \noindent 
  \textit{So, If:}
  \begin{align*}
    \frac{d}{dx}\bigg[ \frac{f(x)}{g(x)}\bigg] = \frac{g(x) \frac{d}{dx}[f(x)] - f(x) \frac{d}{dx}[g(x)]}{[g(x)]^2} 
  .\end{align*}

  \bigbreak \noindent 
  \textit{And:}
  \begin{align*}
    f(x) = e^x \rightarrow f ^{\prime}(x) = e^x \\ 
    g(x) = 1 + x \rightarrow g ^{\prime}(x) = 1
  .\end{align*}
  \bigbreak \noindent 
  \textit{Then:}
  \begin{align*}
    y ^{\prime} = \frac{(1+x)e^x - e^x(1)}{(1+x)^2}\\
    = \frac{e^x + xe^x - e^x}{(1+x)^2} \\ 
    = \frac{xe^x}{(1+x)^2}
  .\end{align*}

  \pagebreak \bigbreak \noindent
  \begin{mdframed}
    \textbf{Example:} Differentiate The Following Function: \textbf{(Product Rule)}
  \end{mdframed}
  \bigbreak \noindent 
  \textbf{2.)} $R(t) = (t+e^t)(3- \sqrt{t})$
  \bigbreak \noindent 
  \textit{So If:}
  \begin{align*}
    \frac{d}{dx}[f(x) \cdot g(x)] = f(x) \frac{d}{dx}[g(x)] + g(x) \frac{d}{dx}[f(x)]
  .\end{align*}
  \bigbreak \noindent 
  \textit{And:}
  \begin{align*}
    f(x) = (t+e^t) \longrightarrow f ^{\prime}(x) = (1+e^t) \\ 
    g(x) = (3-t^{ \frac{1}{2}}) \longrightarrow g ^{\prime}(x) = (0 - \frac{1}{2}t^{- \frac{1}{2}})
  .\end{align*}
  \bigbreak \noindent 
  \textit{Then:}
  \begin{align*}
    R ^{\prime}(t) = (t+e^t)(0- \frac{1}{2}t^{-\frac{1}{2}}) + (1+e^t)(3-t^{-\frac{1}{2}})
  .\end{align*}

  \bigbreak \noindent 
  \textit{Cleanup:}
  \begin{align*}
    R ^{\prime}(t) =- \frac{1}{2}t^{ \frac{1}{2}} - \frac{1}{2} e^t t^{- \frac{1}{2}} + 3 - t^{- \frac{1}{2}} + 3e^t \cdot t^{ \frac{1}{2}} \\ 
    = -\frac{3}{2}t^{ \frac{1}{2}} - \frac{1}{2} e^t t^{- \frac{1}{2}} + 3 + 3e^t \cdot t^{ \frac{1}{2}} \\  
    = -\frac{3}{2}t^{ \frac{1}{2}} - \frac{e^t}{2t^{ \frac{1}{2}}}+ 3 + 3e^t \cdot t^{ \frac{1}{2}}  
  .\end{align*}

  \bigbreak \noindent 
  \textit{Explanation for cleanup:}
  \bigbreak \noindent 
  for the second equation, we just combined like terms, then for the \textbf{\textit{third equation}}, we rewrote the term with the negative power.

  \bigbreak \noindent 
  \begin{mdframed}
    \textbf{Example:} Differentiate the following function \textbf{(Product Rule:)}
  \end{mdframed}
  \bigbreak \noindent 
  \textbf{3.)} $g(x) = 5e^x \sqrt{x}$
  \bigbreak \noindent 
  \textit{So:}
  \begin{align*}
    g ^{\prime}(x) = (5e^x)( \frac{1}{2}x^{- \frac{1}{2}}) + (5e^x)(x^{ \frac{1}{2}})
  .\end{align*}
  \bigbreak \noindent 
  \textit{From here we can simplify by pulling out common factor, $5e^x x^{- \frac{1}{2}}$}
  \bigbreak \noindent 
  \textit{So:}
  \begin{align*}
    5e^xx^{- \frac{1}{2}}( \frac{1}{2} + x^1) \\ 
    = \frac{5e^x}{x^{ \frac{1}{2}}} \cdot \frac{1+2x}{2} \\
    = \frac{5e^x(1+2x)}{2x^{ \frac{1}{2}}}
  .\end{align*}

  \pagebreak \bigbreak \noindent
  \begin{mdframed}
    \textbf{Example:} find $f^{\prime}(x)$ and $f ^{\prime\prime}(x)$
  \end{mdframed}

  \bigbreak \noindent 
  \textbf{1.)} $ f(x ) = x^8e^x$
  \bigbreak \noindent 
  \textit{So:}
  \begin{align*}
    f ^{\prime}(x) = x^8 \cdot e^x + 8x^7 \cdot e^x 
  .\end{align*}
  \bigbreak \noindent 
  \textit{We can factor out an $e^x$}
  \bigbreak \noindent 
  \textit{So, $f ^{\prime}(x)$ is:}
  \begin{align*}
    f ^{\prime}(x) = e^x(x^8+8x^7)
  .\end{align*}
  
  \bigbreak \noindent 
  \textit{Now:}
  \begin{align*}
    f ^{\prime\prime}(x) = e^x(8x^7+56x^6) + (x^8+8x^7)(e^x) \\ 
    = e^x(x^8 + 8x^7 + 8x^7 + 56x^6) \\ 
    = e^x(x^8 + 16x^7 + 56x^6)
  .\end{align*}

  \bigbreak \noindent 
  \begin{mdframed}
    \textbf{Example:} Differentiate (\textbf{\textit{Quotient Rule}}):
    \begin{align*}
      y = \frac{x+1}{x^3+x-2}
    .\end{align*}
  \end{mdframed}
  \bigbreak \noindent 
  \textit{If:}
  \begin{align*}
    \frac{d}{dx}\bigg[ \frac{f(x)}{g(x)}\bigg] = \frac{g(x) \frac{d}{dx}[f(x)] - f(x) \frac{d}{dx}[g(x)]}{[g(x)]^2} 
  .\end{align*}
  \bigbreak \noindent 
  \textit{And:}
  \begin{align*}
    f(x) = x+1 \longrightarrow f ^{\prime}(x) = 1 \\ 
    and \\
    g(x) = x^3 + x -2 \longrightarrow g ^{\prime}(x) = 3x^2+1
  .\end{align*}
  \bigbreak \noindent 
  \textit{Then:}
  \begin{align*}
    y ^{\prime} = \frac{(x^3+x-2)(1) - (x+1)(3x^2+1)}{(x^3+x-2)^2} \\ 
    =  \frac{x^3+x-2 -(3x^3+x+3x^2+1)}{(x^3+x-2)^2} \\
    = \frac{x^3+x-2 -3x^3-x-3x^2-1)}{(x^3+x-2)^2} \\ 
    = \frac{-2x^3-3x^2-3}{(x^3+x-2)^2}
  .\end{align*}

  \pagebreak \bigbreak \noindent
  \begin{mdframed}
    \textbf{Example:} Find the equation of the tangent line and the normal line to the curve $y = \frac{ \sqrt{x}}{x+1}$ at (4,0.4)
  \end{mdframed}

  \bigbreak \noindent 
  \textit{If:}
  \begin{align*}
    \frac{d}{dx}\bigg[ \frac{f(x)}{g(x)}\bigg] = \frac{g(x) \frac{d}{dx}[f(x)] - f(x) \frac{d}{dx}[g(x)]}{[g(x)]^2} 
  .\end{align*}
  \bigbreak \noindent 
  \textit{And:}
  \begin{align*}
    f(x) = x ^{\frac{1}{2}} \longrightarrow f ^{\prime}(x) = \frac{1}{2}x ^{-\frac{1}{2}} \\
    and \\
    g(x) = x+1 \longrightarrow g ^{\prime}(x) = 1
  .\end{align*}
  \bigbreak \noindent 
  \textit{Then:}
  \begin{align*}
    y ^{\prime} = \frac{(x+1)( \frac{1}{2}x ^{-\frac{1}{2}}) - (x ^{\frac{1}{2}})(1)}{(x+1)^2}
  .\end{align*}
  \bigbreak \noindent 
  \textit{Now $m_{tan}$}
  \begin{align*}
    m_{tan} = \frac{(4+1)( \frac{1}{2} \cdot 4 ^{-\frac{1}{2}}) - (4 ^{\frac{1}{2}})}{(x+1)^2} \\ 
    = \frac{5 \cdot \frac{1}{4} - 2}{25} 
  .\end{align*}
  \bigbreak \noindent 
  \textit{We want to multiply by the lcd \textbf{\textit{4}} to clear out the complex fraction}
  \begin{align*}
    \frac{(\frac{5}{4} - 2) \cdot 4}{25 \cdot 4} \\ 
    = \frac{5-8}{100} \\ 
    = - \frac{3}{100}
  .\end{align*}
  \bigbreak \noindent 
  Now to find $m_{normal}$, we take the Reciprical of $m_{tan}$ and change the sign:
  \begin{align*}
    m_{norm} = \frac{100}{3}
  .\end{align*}
  \bigbreak \noindent 
  Now we want to find the equations:
  \begin{center}
    \textbf{Tangent Line:}
  \end{center}
  \begin{align*}
    y - 0.4 = - 0.03(x-4) \\ 
    y-0.4 = -0.03x+0.12 \\ 
    y = -0.03x+0.52
  .\end{align*}
  \begin{center}
    \textbf{Normal Line:}
  \end{center}
  \begin{align*}
    y- \frac{2}{5} = \frac{100}{3}(x-4) \\ 
    y - \frac{2}{5} = \frac{100}{3}x - \frac{400}{3} \\ 
    y = \frac{100}{3}x - \frac{1994}{15}
  .\end{align*}
  Since $\frac{100}{3}$ is a repeating decimal, we stayed in fraction form.

  \pagebreak \bigbreak \noindent
  \begin{Large}
      \begin{mdframed}
          \begin{center}
              \textbf{3.3}
          \end{center}
      \end{mdframed}
  \end{Large}
  \begin{Large}
      \begin{center}
          \textbf{Derivatives of Trigonometric Functions}
      \end{center}
  \end{Large}
  \line(1,0){490}
  
  \bigbreak \noindent \bigbreak \noindent \bigbreak \noindent 
  \begin{mdframed}
    \textbf{Pythagorn Identites:}
    \begin{itemize}
      \item $\sin^{2}{\theta} = 1-\cos^{2}{\theta}$
      \item $\cos^{2}{\theta} = 1 - \sin^{2}{\theta }$
      \item $\sin^{2}{\theta}+\cos^{2}{\theta}= 1$
    \end{itemize}
  \end{mdframed}

  \bigbreak \noindent 
  \begin{mdframed}
    \textbf{2 Limit Formulas:}
    \begin{align*}
      \lim_{\theta \to 0}{ \frac{\sin{\theta}}{\theta}} = 1
    .\end{align*}
    \bigbreak \noindent 
    \begin{center}
      And:
    \end{center}
    \begin{align*}
      \lim_{\theta \to 0}{ \frac{\cos{\theta} -1} {\theta}} = 0
    .\end{align*}
  \end{mdframed}
  \bigbreak \noindent 
 \begin{mdframed}
   \textbf{Lets Derive $ \frac{d}{dx}(\sin{x})$}:
   \begin{align*}
     \frac{d}{dx}(\sin{x}) = \lim_{h \to 0}{ \frac{\sin(x+h) - \sin{x}}{h}}
   .\end{align*}
 \end{mdframed} 
 \bigbreak \noindent 
 We will refer back to the formula for $\sin{(a+b)}$ $\rightarrow$ $ \sin{A} \cos{B} + \cos{A} \sin{B} $ to expand $ \sin{(x+h)}$
 \bigbreak \noindent 
  \textit{So:}
  \begin{align*}
    \lim_{h \to 0}{ \frac{ \sin{x} \cos{h} + \cos{x} \sin{h} - \sin{x}}{h}} 
  .\end{align*}
  \bigbreak \noindent 
  \textit{We are going to split this equation:}
  \begin{align*}
    \lim_{h \to 0}{ \frac{ \sin{x} \cos{h} - \sin{x}}{h}} + \lim_{h \to 0}{ \frac{ \cos{x} \cdot \sin{h}}{h}}
  .\end{align*}
  \bigbreak \noindent 
  \textit{Since $ \sin{x}$ and $ \cos{x}$ is not changing, it is therefore a constant and we can do the following:}
  \begin{align*}
  (\sin{x}) \bigg(\lim_{h \to 0}{ \frac{ \cos{h}-1}{h}}\bigg) + (\cos{x}) \bigg( \lim_{h \to 0}{ \frac{\sin{h}}{h}}\bigg)
  .\end{align*}
  \bigbreak \noindent 
  \textit{Now we can use the formulas above and we are left with:}
  \begin{align*}
   0 + \cos{x} \cdot 1 \\
   = \cos{x}
  .\end{align*}
  \bigbreak \noindent 
  \begin{mdframed}
    \textbf{Summary:}
    \begin{align*}
      \frac{d}{dx} \sin{x} = \cos{x}
    .\end{align*}
  \end{mdframed}

  \bigbreak \noindent 
   \begin{mdframed}
     \textbf{Lets Derive $ \frac{d}{dx}(\cos{x})$}:
     \begin{align*}
       \frac{d}{dx}(\cos{x}) = \lim_{h \to 0}{ \frac{\cos(x+h) - \cos{x}}{h}}
     .\end{align*}
   \end{mdframed} 
   \bigbreak \noindent  
 We will refer back to the formula for $\cos{(A+B)} \rightarrow \cos{A} \cos{B} - \sin{A} \sin{B}$ to expand $ \cos{(x+h)}$
 \bigbreak \noindent 
 \textit{So:} 
 \begin{align*}
   \lim_{h \to 0}{ \frac{ \cos{x} \cos{h} - \sin{x} \sin{h} - \cos{x}}{h}}
 .\end{align*}
 \bigbreak \noindent 
  \textit{Just like the one above, we are going to group the terms that have x:}
  \begin{align*}
    \lim_{h \to 0}{ \frac{ \cos{x} \cos{h} - \cos{x}}{h}} - \lim_{h \to 0}{ \frac{ \sin{x} \sin{h}}{h}}
  .\end{align*}
  \bigbreak \noindent 
  \textit{Now we pull out the constants:}
  \begin{align*}
    ( \cos{x}) \bigg( \lim_{h \to 0}{ \frac{ \cos{h} - 1}{h}}\bigg) - ( \sin{x}) \bigg( \lim_{h \to 0}{ \frac{ \sin{h}}{h}}\bigg)
  .\end{align*}
  \bigbreak \noindent 
  \textit{Now if we use the fomulas listed at the start of this section we are left with:}
  \begin{align*}
    ( \cos{x})(0) - ( \sin{x})(1) \\ 
    = - \sin{x}
  .\end{align*}

  \bigbreak \noindent \bigbreak \noindent 
  \begin{mdframed}
  \begin{large}
    \textbf{Deriviatives of Trigonometric Functions:}
  \end{large}
  \bigbreak \noindent 
  \begin{itemize}
    \item $ \frac{d}{dx}( \sin{x}) = \cos{x}$
    \item $ \frac{d}{dx}( \cos{x}) = - \sin{x}$
    \item $ \frac{d}{dx}( \tan{x}) = \sec^2{x}$
    \item $ \frac{d}{dx}( \csc{x}) =-\csc{x}\cot{x}$
    \item $ \frac{d}{dx}( \sec{x}) =\sec{x}\tan{x}$
    \item $ \frac{d}{dx}( \cot{x}) =-\csc^2{x}$
  \end{itemize}
  \end{mdframed}

  \bigbreak \noindent 
  \begin{mdframed}
    \textbf{Examples: Differentiate:}
    \begin{align*}
      f(x) = \sqrt{x} \sin{x}
    .\end{align*}
  \end{mdframed}
  \bigbreak \noindent 
  \textit{If:}
  \begin{align*}
    \frac{d}{dx}[f(x) \cdot g(x)] = f(x) \frac{d}{dx}[g(x)] + g(x) \frac{d}{dx}[f(x)]
  .\end{align*}
  \bigbreak \noindent 
  \textit{And:}
  \begin{align*}
    f(x) = x^{ \frac{1}{2}} \\ 
    g(x) = \sin{x}
  .\end{align*}
  \begin{align*}
    f ^{\prime}(x) = \frac{1}{2}x^{-\frac{1}{2}} \\
    g ^{\prime}(x) = \cos{x}
  .\end{align*}
  \bigbreak \noindent 
  \textit{Then:}
  \begin{align*}
    f ^{\prime}(x) = x^{ \frac{1}{2}} \cdot \cos{x} + \sin{x} \cdot \frac{1}{2}x^{ - \frac{1}{2}} \\ 
    \frac{1}{2}x^{- \frac{1}{2}}(2x \cdot \cos{x} + \sin{x}) \\
    = \frac{2x \cdot \cos{x} + \sin{x}}{2x^{ \frac{1}{2}}}
  .\end{align*}

  \bigbreak \noindent 
  \begin{mdframed}
    \textbf{Example: Differentiate:}
    \begin{align*}
      g(t) = 4 \sec{t} + \tan{t} 
    .\end{align*}
  \end{mdframed}
  \bigbreak \noindent 
  \textit{So:}
  \begin{align*}
    g ^{\prime}(t) = 4 \cdot \sec{t} \tan{t} + \sec^2{t} \\
    = 4 \cdot \frac{1}{ \cos{t}} \cdot \frac{ \sin{t}}{ \cos{t}} + \frac{1}{ \cos^2{t}} \\
    = 4 \cdot \frac{ \sin{t}}{\cos^2{t}} + \frac{1}{ \cos^2{t}} \\
    = \frac{4 \sin{t}+1}{ \cos^2{t}}
  .\end{align*}

  \bigbreak \noindent 
  \begin{mdframed}
    \textbf{Example:}
    \begin{align*}
      y=  \frac{1- \sec{x}}{ \tan{x}}
    .\end{align*}
  \end{mdframed}
  \textit{If:}
  \begin{align*}
    \frac{d}{dx}\bigg[ \frac{f(x)}{g(x)}\bigg] = \frac{g(x) \frac{d}{dx}[f(x)] - f(x) \frac{d}{dx}[g(x)]}{[g(x)]^2}
  .\end{align*}
  \textit{And:}
  \begin{align*}
    f(x) = 1- \sec{x} \\
    g(x) = \tan{x}
  .\end{align*}
  \begin{align*}
    f ^{\prime}(x) = \sec{x} \tan{x} \\ 
    g ^{\prime}(x) = \sec^2{x}
  .\end{align*}
  \textit{Then:}
  \begin{align*}
    y ^{\prime} = \frac{( \tan{x})(- \sec{x} \tan{x}) - (1 - \sec{x})( \sec^2{x})}{ \tan^2{x} } \\ 
    =  \frac{ - \sec{x} \tan^2{x}- (\sec^2{x} - \sec^3{x})}{\tan^2{x}} \\ 
    =  \frac{ - \sec{x} \tan^2{x}-\sec^2{x} + \sec^3{x}}{\tan^2{x}} \\ 
    = \frac{- \frac{1}{ \cos{x}} \cdot \frac{ \sin^2{x}}{ \cos^2{x}} - \frac{1}{ \cos^2{x}} + \frac{1}{ \cos^3{x}}}{ \frac{ \sin^2{x}}{ \cos^2{x}}}
  .\end{align*}
  \textit{We need to multiply by the lcd $ \cos^3{x}$:}
  \begin{align*}
    \frac{- \sin^2{x} - \cos{x} + 1}{ \sin^2{x} \cos{x}} 
  .\end{align*}
  \bigbreak \noindent 
  \textit{In the numerator we notice we have $ 1- \sin^2{x}$, which is equal to $ \cos^2{x}$, so:}
  \begin{align*}
    \frac{ \cos^2{x} - \cos{x}}{ \sin^2{x} \cos{x}} \\ 
    = \frac{ \cos{x}( \cos{x} - 1)}{ \sin^2{x} \cos{x}} \\ 
    = \frac{ \cos{x} -1}{ \sin^2{x}}
  .\end{align*}
  \bigbreak \noindent 
  \textit{And we can replace the denominator with $ 1- \cos^2{x}$:}
  \begin{align*}
    \frac{ \cos{x} -1}{1 - \cos^2{x}}
  .\end{align*}
  \bigbreak \noindent 
  \textit{And we notice that the denominator is a difference of squares, so we can factor it into:}
  \begin{align*}
    \frac{ \cos{x} -1}{(1 - \cos{x})(1 + \cos{x})} \\ 
    = \frac{- (1 - \cos{x})}{(1 - \cos{x})(1 + \cos{x})} \\ 
    = \frac{- 1}{1 + \cos{x}}
  .\end{align*}

  \bigbreak \noindent 
  \begin{large}
    \textbf{Limits:}
  \end{large}
  \begin{mdframed}
    \textbf{Recall:}
    \begin{align*}
      \lim_{\theta \to 0}{ \frac{ \sin{\theta}}{\theta}} = 1\ and\ \lim_{\theta \to 0}{ \frac{\theta}{ \sin{\theta}} = 1}
    .\end{align*}
    \begin{center}
      Also:
    \end{center}
    \begin{align*}
      \lim_{\theta \to 0}{ \frac{ \cos{\theta} - 1}{\theta}} = 0
    .\end{align*}
  \end{mdframed}
  \bigbreak \noindent 
  \begin{mdframed}
    \textbf{Example: Find the Limit:}
    \begin{align*}
      \lim_{x \to 0}{ \frac{ \sin{4x}}{ \sin{6x}}}  
    .\end{align*}
  \end{mdframed}
  \bigbreak \noindent 
  \textit{We want to be able to use the fomulas above, so we do:}
  \bigbreak \noindent 
  \begin{align*}
    \lim_{x \to 0}{ \frac{ \sin{4x}}{4x}} \cdot \frac{4x}{1} \cdot \frac{6x}{ \sin{6x}} \cdot \frac{1}{6x} \\
    = 1 \cdot 4 \cdot 1 \cdot \frac{1}{6} \\ 
    = \frac{2}{3}
  .\end{align*}

  \bigbreak \noindent 
  \begin{mdframed}
    \textbf{Example: Find the Limit:}
    \begin{align*}
      \lim_{\theta \to 0}{ \frac{ \cos{\theta} -1}{ \sin{\theta}}}
    .\end{align*}
  \end{mdframed}
  \bigbreak \noindent 
  \textit{To exercise the formulas above, we will rewrite as:}
  \begin{align*}
    \lim_{\theta \to 0}{ \frac{ \cos{\theta} - 1}{ \sin{\theta}}} \cdot \frac{\theta}{\theta} \\ 
    = \frac{ \cos{\theta} - 1}{\theta} \cdot \frac{\theta}{ \sin{\theta}} \\
    = 0 \cdot 1 \\
    = 0
  .\end{align*}

  \bigbreak \noindent 
  \begin{mdframed}
    \textbf{Example: Find the Limit:}
    \begin{align*}
      \lim_{t \to 0}{ \frac{ \sin^2{3t}}{t^2}}
    .\end{align*}
  \end{mdframed}
  \textit{We rewrite as:}
  \begin{align*}
    \lim_{t \to 0}{ \bigg( \frac{ \sin{3t}}{t}\bigg)^2} \\
    = \bigg( \frac{ \sin{3t}}{3t} \cdot \frac{3}{1}\bigg)^2 \\
    = 1 \cdot 3^2 \\ 
    = 9
  .\end{align*}
  
  \pagebreak \bigbreak \noindent
  \begin{Large}
      \begin{mdframed}
          \begin{center}
              \textbf{3.4}
          \end{center}
      \end{mdframed}
  \end{Large}
  \begin{Large}
      \begin{center}
        \textbf{The Chain Rule / Differentiation Examples using the Product, Quotient, and Chain Rules}
      \end{center}
  \end{Large}
  \line(1,0){490}

  \bigbreak \noindent \bigbreak \noindent 
  \begin{Large}
    \textbf{The Chain Rule:}
  \end{Large}
  \bigbreak \noindent 
  We will use the chain rule to find Deriviatives of composite functions.

  \bigbreak \noindent 
  \begin{mdframed}
    \textbf{Example: Find the derivative of }
    \begin{align*}
      F(x) = \sqrt{4+3x}
    .\end{align*}
  \end{mdframed}
  \bigbreak \noindent 
  F(x) is a composite function made up of:
  \begin{align*}
    g(x) = 4 +3x \\
    and \\
    f(x) = \sqrt{x}
  .\end{align*}
  \bigbreak \noindent 
  \textbf{\textit{\underline{Therefore:}}}
  \begin{align*}
    F(x) = f(g(x))
  .\end{align*}
  \bigbreak \noindent 
  \textbf{\textit{\underline{Process:}}} 
  \bigbreak \noindent 
  \textit{Let:}
  \begin{align*}
    u = g(x) = 4+3x
  .\end{align*}
  \bigbreak \noindent 
  \textit{Then:}
  \begin{align*}
    F(x) = f(u)\ and\ F^{\prime}(x) = f^{\prime}(u) \cdot g^{\prime}(x)
  .\end{align*}

  \bigbreak \noindent \bigbreak \noindent 
  \begin{Large}
    \textbf{The Chain Rule (2):}
  \end{Large}
  \bigbreak \noindent 
  If F(x) = f(g(x)), then:
  \begin{align*}
    F^{\prime}(x) = f^{\prime}(g(x)) \cdot g^{\prime}(x)
  .\end{align*}
  \begin{center}
    Or:
  \end{center}
  \bigbreak \noindent 
  If y = f(u) = f(g(x)), then:
  \begin{align*}
    \frac{dy}{dx} = \frac{dy}{dx} \cdot \frac{du}{dx}
  .\end{align*}

  \bigbreak \noindent 
  \begin{mdframed}
    \textbf{Example:} Find the Derivative:
    \begin{align*}
      f(x) = (1+x^{4})^{\frac{2}{3}}
    .\end{align*}
  \end{mdframed}
  \bigbreak \noindent 
  \textit{So:}
  \begin{align*}
    f^{\prime}(x)=\frac{2}{3}(1+x^{4})^{-\frac{1}{3}} \cdot (4x^{3}) \\ 
    = \frac{8x^{3}}{3(1+x^{4})^{\frac{1}{3}}}
  .\end{align*}

  \bigbreak \noindent 
  \begin{mdframed}
    \textbf{Example: Differentiate the following function:}
    \begin{align*}
      f(t) = \sqrt[3]{1+\tan{t}}
    .\end{align*}
  \end{mdframed}
  \bigbreak \noindent 
  \textit{So:}
  \begin{align*}
    f(t) = (1+\tan{t})^{\frac{1}{3}}
  .\end{align*}
  \bigbreak \noindent 
  \textit{Now:}
  \begin{align*}
    f^{\prime}(t) = \frac{1}{3}(1+\tan{t})^{-\frac{2}{3}} \cdot (\sec^2{t}) \\ 
    = \frac{\sec^{2}{t}}{3(1+\tan{t})^{\frac{2}{3}}}
  .\end{align*}

  \bigbreak \noindent 
  \begin{mdframed}
    \textbf{Example: Differentiate The following function:}
    \begin{align*}
      y = (x^{2}+1)(\sqrt[3]{x^{2}+2})
    .\end{align*}
  \end{mdframed}
  \bigbreak \noindent
  \textit{So:}
  \begin{align*}
   y = (x^{2} +1)(x^{2}+2)^{\frac{1}{3}} 
  \end{align*}
  \bigbreak \noindent
  \textit{Now:}
  \begin{align*}
    \frac{dy}{dx} = (x^2+1)[\frac{1}{3}(x^2+2)^{-\frac{2}{3}}(2x)
  \end{align*}




















  
  \pagebreak \bigbreak \noindent
  \begin{mdframed}
    \textbf{Recall:}
    \bigbreak \noindent 
    \begin{center}
      Product Rule:
    \end{center}
    \begin{align*}
      \frac{d}{dx}[f(x) \cdot g(x)] = f(x) \frac{d}{dx}[g(x)] + g(x) \frac{d}{dx}[f(x)]
    .\end{align*}
    \bigbreak \noindent 
    \begin{center}
      Quotient Rule:
    \end{center}
    \begin{align*}
      \frac{d}{dx}\bigg[ \frac{f(x)}{g(x)}\bigg] = \frac{g(x) \frac{d}{dx}[f(x)] - f(x) \frac{d}{dx}[g(x)]}{[g(x)]^2}
    .\end{align*}
  \end{mdframed}
  
  \bigbreak \noindent 
  \begin{mdframed}
    \textbf{The Chain Rule:}
    \bigbreak \noindent 
    If F(x) = f(g(x)), then:
    \begin{align*}
      F ^{\prime}(x) = f ^{\prime}(g(x)) \cdot g ^{\prime}(x)
    .\end{align*}
    \begin{center}
      Or:
    \end{center}
    \bigbreak \noindent 
    If f(u) = f(g(x)), then:
    \begin{align*}
      \frac{dy}{dx} = \frac{dy}{du} \cdot \frac{du}{dx}
    .\end{align*}
  \end{mdframed}
  \bigbreak \noindent 
  \begin{mdframed}
    \textbf{Example:} Differentiate the following function:
    \begin{align*}
      r = \frac{ \sqrt{\theta} -3}{ \sqrt{\theta} +3}
    .\end{align*}
  \end{mdframed}
  \bigbreak \noindent 
  \textit{If:}
  \begin{align*}
    \frac{d}{dx}\bigg[ \frac{f(x)}{g(x)}\bigg] = \frac{g(x) \frac{d}{dx}[f(x)] - f(x) \frac{d}{dx}[g(x)]}{[g(x)]^2}
  .\end{align*}
  \bigbreak \noindent 
  \textit{And:}
  \begin{align*}
    f(x) = \sqrt{ \theta } -3 \\
    f ^{\prime}(x) = \frac{1}{2} \theta^{- \frac{1}{2}}
  .\end{align*}
  \begin{align*}
    g(x) = \sqrt{ \theta } + 3 \\ 
    g ^{\prime}(x) = \frac{1}{2} \theta^{- \frac{1}{2}}
  .\end{align*}
  \bigbreak \noindent 
  \textit{Then:}
  \begin{align*}
    \frac{dr}{d\theta} = \frac{( \sqrt{ \theta } +3)( \frac{1}{2} \theta^{- \frac{1}{2}}) - ( \sqrt{ \theta } -3)( \frac{1}{2} \theta ^{-\frac{1}{2}})}{ ( \sqrt{ \theta } + 3)^2} \\ 
    = \frac{ \frac{1}{2} \theta^{-\frac{1}{2}}( \sqrt{ \theta }+3 - \sqrt{ \theta } +3)}{(\sqrt{ \theta } +3)^2} \\ 
    = \frac{\frac{1}{2} \theta^{-\frac{1}{2}}(6)}{(\sqrt{ \theta } +3)^2} \\
    = \frac{3 \cdot  \theta ^{-\frac{1}{2}}}{(\sqrt{ \theta } +3)^2} \\ 
    = \frac{3}{ \sqrt{ \theta }(\sqrt{ \theta } +3)^2}
  .\end{align*}
  \bigbreak \noindent 
  \nt{It's fine that we have a radical in the denominator because there was one in the original equation.}

  \bigbreak \noindent 
  \begin{mdframed}
    \textbf{Example: Differentiate the following function:}
    \begin{align*}
      p = \frac{4 + \sec{q}}{4 - \sec{q}}
    .\end{align*}
  \end{mdframed}
  \bigbreak \noindent 
  \textit{We will rewrite in terms of \textbf{\textit{sin and cos}}} 
  \begin{align*}
    p = \frac{4 + \frac{1}{ \cos{q}}}{4 - \frac{1}{ \cos{q}}} 
  .\end{align*}
  \bigbreak \noindent 
  \textit{Now find common denominator to clear out fractions ($ \cos{q}$)}:
  \begin{align*}
    p= \frac{4 \cos{q} + 1}{4 \cos{q} - 1} 
  .\end{align*}
  \bigbreak \noindent 
  \textit{Now we differentiate:}
  \begin{align*}
    f(x) = 4 \cos{q} + 1 \\
    f ^{\prime}(x) = -4 \sin{q}
  .\end{align*}
  \begin{align*}
    g(x) = 4 \cos{q} = 1 \\ 
    g ^{\prime}(x) = -4 \sin{q}
  .\end{align*}
  \bigbreak \noindent 
  \textit{Now plug into Quotient Rule:}
  \begin{align*}
    \frac{dp}{dq} = \frac{(4 \cos{q} - 1)(-4 \sin{q}) - (4 \cos{q} + 1)(-4 \sin{q})}{(4 \cos{q}-1)^2 }
  .\end{align*}
  \bigbreak \noindent 
  \textit{we see we can factor out an $-4 \sin{q}$}:
  \begin{align*}
    \frac{dp}{dq} = \frac{-4 \sin{q}(4 \cos{q} -1)-(4 \cos{q} +1)}{(4 \cos{q}-1)^2 } \\ 
    = \frac{-4 \sin{q}(4 \cos{q} -1-4 \cos{q} -1)}{(4 \cos{q}-1)^2 } \\ 
    = \frac{-4 \sin{q}(-2)}{(4 \cos{q}-1)^2 } \\
    = \frac{-8 \sin{q}}{(4 \cos{q}-1)^2 }
  .\end{align*}

  \bigbreak \noindent 
  \begin{mdframed}
    \textbf{Example: Differentiate the following function:}
    \begin{align*}
      h(x) = \bigg( \frac{ \cos{x}}{1+ \sin{x}}\bigg)^4
    .\end{align*}
  \end{mdframed}
  \bigbreak \noindent 
  \textit{First lets figure out our Deriviatives from whats withing the parenthesis:}
  \begin{align*}
    f(x) = \cos{x} \\
    f ^{\prime}(x) = - \sin{x}
  .\end{align*}
  \begin{align*}
    g(x) = 1+ \sin{x} \\ 
    g ^{\prime}(x) =  \cos{x}
  .\end{align*}
  \bigbreak \noindent 
  \textit{We will start by using the power rule and the chain rule with the quotient rule:}
  \begin{align*}
    h ^{\prime}(x) = 4 \bigg[\frac{\cos{x}}{1+\sin{x}}\bigg]^3  \cdot \bigg[ \frac{(1+ \sin{x})(- \sin{x}) - ( \cos{x})( \cos{x})}{(1+ \sin{x})^2}\bigg]
  .\end{align*}
  \bigbreak \noindent \bigbreak \noindent 
  \textit{Now we want to distribute the exponent \textbf{\textit{3}}, into the terms in the numerator and denominator}

  \begin{align*}
   \frac{4\cos^{3}{x}}{(1+\sin{x})^{3}} \cdot \frac{-\sin{x}-\sin^{2}{x}-\cos^{2}{x}}{(1+\sin{x})^2}  \\ 
  .\end{align*}
  \bigbreak \noindent 
  \textit{We are going to factor out a -1 and bring it infront of the 4:}
  \begin{align*}
   \frac{-4\cos^{3}{x}}{(1+\sin{x})^{3}} \cdot \frac{\sin{x}+\sin^{2}{x}+\cos^{2}{x}}{(1+\sin{x})^2}  \\ 
  .\end{align*}
  \bigbreak \noindent 
  \textit{We know that $\sin^{2}{x} + \cos^{2}{x} =1$, so:}
  \begin{align*}
   \frac{-4\cos^{3}{x}}{(1+\sin{x})^{3}} \cdot \frac{\sin{x}+1}{(1+\sin{x})^2}  \\ 
  .\end{align*}
  \bigbreak \noindent 
  \textit{Now we can divide by common factor in the numerator:}
  \begin{align*}
    \frac{-4\cos^{3}{x}}{(1+\sin{x})^{3}} \cdot \frac{1}{1+\sin{x}} \\ 
    = \frac{-4\cos^{3}{x}}{(1+\sin{x})^{4}} 
  .\end{align*}

  \pagebreak \bigbreak \noindent
  \begin{mdframed}
    \textbf{Example: Differentiate the following function:}
    \begin{align*}
      y=  (e^{\cos{(\frac{t}{9})}})^4
    .\end{align*}
  \end{mdframed}
  \bigbreak \noindent 
  \textit{So by using both the product rule and the chain rule, we get:}
  \begin{align*}
    y^{\prime} = 4(e^{\cos{\frac{t}{9}}})^3 \cdot e^{\cos{\frac{t}{9}}}
  .\end{align*}

\end{document}
