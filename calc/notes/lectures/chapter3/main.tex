\documentclass{report}

\input{~/dev/latex/template/preamble.tex}
\input{~/dev/latex/template/macros.tex}

\title{\Huge{Calculus 1: Chapter 3}}
\author{\huge{Nathan Warner}}
\date{\huge{Feb 12, 2023}}

\pgfpagesdeclarelayout{boxed}
{
  \edef\pgfpageoptionborder{0pt}
}
{
  \pgfpagesphysicalpageoptions
  {%
    logical pages=1,%
  }
  \pgfpageslogicalpageoptions{1}
  {
    border code=\pgfsetlinewidth{1.5pt}\pgfstroke,%
    border shrink=\pgfpageoptionborder,%
    resized width=.95\pgfphysicalwidth,%
    resized height=.95\pgfphysicalheight,%
    center=\pgfpoint{.5\pgfphysicalwidth}{.5\pgfphysicalheight}%
  }%
}

\pgfpagesuselayout{boxed}

\begin{document}
    \maketitle
    \begin{center}
        \begin{Huge}
            \textbf{Chapter 3}
        \end{Huge}
    \end{center}
    \line(1,0){490}
    \bigbreak \noindent \bigbreak \noindent  
    \begin{Huge}
        \noindent \textbf{Contents}
    \end{Huge}
    \bigbreak \noindent \bigbreak \noindent 
    \begin{Large}
        \textbf{3.1: Differential Rule}
        \bigbreak \noindent \bigbreak \noindent 
        \textbf{3.2: The Product and Quotient Rule}
        \bigbreak \noindent \bigbreak \noindent 
        \textbf{3.3: Derivatives of Trigonometric Functions}
        \bigbreak \noindent \bigbreak \noindent 
        \textbf{3.4.1: The Chain Rule}
        \bigbreak \noindent \bigbreak \noindent 
        \textbf{3.4.2: Differentiation Examples using the Product, Quotient, and Chain Rules}
        \bigbreak \noindent \bigbreak \noindent 
        \textbf{3.5: Implicit Differentiation and 3.6 (Part 1) Derivatives of Inverse Trigonometric Functions}
        \bigbreak \noindent \bigbreak \noindent 
        \textbf{3.6: (Part 2) Derivatives of Logarithmic Functions}
        \bigbreak \noindent \bigbreak \noindent 
        \textbf{3.7: Rates of Change in the Natural and Social Sciences}
        \bigbreak \noindent \bigbreak \noindent 
        \textbf{3.8: Exponential Growth and Decay -- Newton's Law of Cooling}
        \bigbreak \noindent \bigbreak \noindent 
        \textbf{3.9: Related Rates}
        \bigbreak \noindent \bigbreak \noindent 
        \textbf{3.10: Linear Approximations and Differentials}
        \bigbreak \noindent \bigbreak \noindent 
        \textbf{3.11: Hyperbolic Functions}
    \end{Large}

    \pagebreak \bigbreak \noindent
    \begin{Large}
        \begin{mdframed}
            \begin{center}
                \textbf{3.1}
            \end{center}
        \end{mdframed}
    \end{Large}
    \begin{Large}
        \begin{center}
            \textbf{Differential Rule:}
        \end{center}
    \end{Large}
    \line(1,0){490}
    
    \bigbreak \noindent 
    \begin{mdframed}
        \textbf{\textit{Diffential Fomulas:}}
        \begin{itemize}
            \item $ \frac{d}{dx}(c) = 0$
            \item $ \frac{d}{dx}(x) = 1$
            \item $ \frac{d}{dx}(x^n) = n \cdot x^{n-1} \rightarrow$ \text{\textbf{\textit{Power Rule}}}
            \item $ \frac{d}{dx}[c \cdot f(x)] = c \cdot \frac{d}{dx}[f(x)]$
            \item $ \frac{d}{dx}[f(x) \pm g(x)] = \frac{d}{dx}f(x)\pm \frac{d}{dx}g(x)$
        \end{itemize}
    \end{mdframed}
    \bigbreak \noindent \bigbreak \noindent 
    \ex{}{Differentiate the following functions:} 
    \bigbreak \noindent 
    \begin{mdframed}
       \textbf{1.)} $f(t) = \frac{1}{2}t^6 - 3t^4 +1$ 
    \end{mdframed}
  
   \bigbreak \noindent  \bigbreak \noindent 
   For the first term, we will use the \textbf{\textit{Third and Fourth}} Rule:
   \begin{align*}
      \frac{1}{2} \cdot 6t^{t-1}
   .\end{align*}
   \bigbreak \noindent 
   For the second term, \textit{$-3t^4$}, We will use the \textbf{\textit{Third and Fifth}} Rule:
   \begin{align*}
      -3 \cdot 4t^{4-1} 
   .\end{align*}
   \bigbreak \noindent 
   The last term is a constant, so according to the first rule, the Derivative of a constant is \textbf{\textit{Zero}}:

  \begin{center}
    \textit{So our full equation is:}
  \end{center}
  \begin{align*}
    f\prime(x) = \frac{1}{2} \cdot 6t^{6-1} - 3 \cdot 4t^{4-1} + 0 \\ 
    = 3t^5-12t^3 
  .\end{align*}
  
  \bigbreak \noindent \bigbreak \noindent 
  \begin{mdframed}
    \textbf{2.)} $h(x) = (x-2)(2x+3)$ 
  \end{mdframed}
  First we need to distribute out the terms:
  \begin{align*}
    h(x) = 2x^2 + 3x -4x -6 \\ 
    = 2x^2 -x-6
  .\end{align*}
  \bigbreak \noindent 
  \begin{center}
    Now this is the function we want to diferentiate.
  \end{center}
  \bigbreak \noindent 
  \textit{So $\rightarrow$}
  \begin{align*}
    h\prime(x) = 2 \cdot 2x^{2-1} - 1 - 0  \\ 
    h\prime(x) = 4x -1
  .\end{align*}
  
  \begin{mdframed}
    \textbf{3.)} $y = \frac{x^2 - 2 \sqrt{x}}{x}$
  \end{mdframed}
  \bigbreak \noindent 
  \textit{So:}
  \begin{align*}
      y = \frac{x^2 - 2 x^{\frac{1}{2}}}{x}
  .\end{align*}
  \bigbreak \noindent 
  \begin{center}
    \textit{Since the numerator only has \textbf{one term}, we can split the equation like:}
  \end{center}
  \begin{align*}
    y = \frac{x^2}{x} - \frac{2x^{ \frac{1}{2}}}{x} \\ 
    y = x - 2x^{- \frac{1}{2}}
  .\end{align*}
  \bigbreak \noindent 
  \textit{Now:}
  \begin{align*}
    \frac{dy}{dx} = 1 - 2 \cdot (-\frac{1}{2})x^{- \frac{1}{2} - 1} \\ 
    \frac{dy}{dx} = 1 + x^{- \frac{3}{2}}
  .\end{align*}
  \bigbreak \noindent 
  \textit{And we can even rewrite it as:}
  \begin{align*}
    \frac{dy}{dx} = 1 + \frac{1}{3^{ \frac{3}{2}}}  
  .\end{align*}
  
  \bigbreak \noindent \bigbreak \noindent  
  \begin{mdframed}
    \textbf{4.)} $ V = ( \sqrt{x} + \frac{1}{ \sqrt[3]{x}})^2$
  \end{mdframed}
  \bigbreak \noindent 
  \textit{So:}
  \begin{align*}
    V = (x^{ \frac{1}{2}} + x^{ \frac{-1}{3}})^2 \\ 
    = (x^{ \frac{1}{2}})^2  + 2(x^{ \frac{1}{2}})(x^{- \frac{1}{3}}) + (x^{- \frac{1}{3}})^2 \\
    = x + 2x^{ \frac{1}{6}} + x^{ - \frac{2}{3}}
  .\end{align*}
  \bigbreak \noindent 
  \textit{Now we find the Derivative:}
  \begin{align*}
    V\prime = 1 + 2 \cdot \frac{1}{6} x^{ \frac{1}{6} - 1} + ( \frac{-2}{3})x^{- \frac{2}{3} -1} \\ 
    v\prime = 1 +  \frac{1}{3}x^{- \frac{5}{6}} - \frac{2}{3}x^{- \frac{5}{3}} \\ 
    v\prime = 1 + \frac{1}{3x^{ \frac{5}{6}}} - \frac{2}{3x^{ \frac{5}{3}}}
  .\end{align*}

  \pagebreak \bigbreak \noindent
  \begin{large}
    \textbf{Exponential Functions:}
  \end{large}

  \bigbreak \noindent \bigbreak \noindent 
  \textbf{Recall:} $(1+ \frac{1}{n})^n \rightarrow e \approx 2.71828... as n \rightarrow \infty $

  \bigbreak \noindent \bigbreak \noindent  
  \dfn{Definiton of e:}{ $\lim\limits_{h \to 0}{ \frac{e^h -1}{h}} = 1$}
  \bigbreak \noindent 
  \nt{\textit{We'll use the above definiton to derive $ \frac{d}{dx}(e^x)$}}

  \bigbreak \noindent 
  $\rightarrow$ Let $f(x) = e^x$
  \begin{align*}
    f\prime(x) = \lim\limits_{h \to 0}{ \frac{f(x+h) - f(x)}{h}}
  .\end{align*}
\end{document}
