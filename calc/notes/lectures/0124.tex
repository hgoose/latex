\documentclass{report}

\input{~/dev/latex/template/preamble.tex}
\input{~/dev/latex/template/macros.tex}

\title{\Huge{Calculus 1 Notes}}
\author{\huge{Nathan Warner}}
\date{\huge{December 17, 2023}}
\graphicspath{{./}}

\begin{document}
    \maketitle

    \begin{center}
        \Huge{Chapter 2}
    \end{center}
    \line(1,0){470} 

    \bigbreak \noindent \bigbreak \noindent  
    \begin{Huge}
        \noindent \textbf{Contents}
    \end{Huge}
    
    \bigbreak \noindent \bigbreak \noindent 
    \begin{Large}
        \textbf{2.1: The Tangent and Velocity Problems }
        \bigbreak \noindent \bigbreak \noindent  
        \textbf{2.2.1 The Limits of a Function }
        \bigbreak \noindent \bigbreak \noindent  
        \textbf{2.2.2 Infinite Limits }
        \bigbreak \noindent \bigbreak \noindent 
        \textbf{2.2.3 Finding Limits of a Trigonometric Function }
        \bigbreak \noindent \bigbreak \noindent 
        \textbf{2.3 Calculating Limits Using Limit Laws}
        \bigbreak \noindent \bigbreak \noindent 
        \textbf{2.5.1 Continuity }
        \bigbreak \noindent \bigbreak \noindent 
        \textbf{2.5.2 One-Sided Continuity}
        \bigbreak \noindent \bigbreak \noindent 
        \textbf{2.6 Limits at Infinity: Horizontal Asymptotes}
        \bigbreak \noindent \bigbreak \noindent  
        \textbf{2.7 Derivatives and Rates of Change}
        \bigbreak \noindent \bigbreak \noindent 
        \textbf{2.8.1 The Derivative of a Function}
        \bigbreak \noindent \bigbreak \noindent 
        \textbf{2.8.2 Finding The Derivatives Using The Limit Definition}
    \end{Large}

    \pagebreak
    \begin{Large}
        \noindent \textbf{2.1: The Tangent and Velocity Problems}
    \end{Large}

    \bigbreak \noindent \bigbreak \noindent \bigbreak \noindent 
    \begin{large}
       \noindent \textbf{The Tangent Problem: } 
    \end{large}
   
    \bigbreak \noindent 
    \qs{}{Can we find an equation of the tangent line to $y=x^2$ at the point P(1,1)?}
   
    \bigbreak \noindent 
    \begin{center}
        \includegraphics[scale=0.8]{1.png}    
    \end{center}
    
    \pf{Explanation}{. \\
        $y=x^2$: Red parabola \\
        Tangent line: Blue line \\ 
        Secent Line: Pink line with points q and p
    }

    We are asked to get the equation of the tangent line to $y=x^2$ at the point P(1,1), 
    However to find the equation of this line we know we need \textbf{2 things,} 
    \begin{itemize}
        \item Point
        \item Slope
    \end{itemize}

    \noindent Since we only have one point, we cannot find slope. Therefore, we must use 
    another point as an approximation and create a secent line instead. \textbf{This secent line is 
    the pink line in the above graphic.}
    
    \bigbreak \noindent 
    \textbf{So}, lets use the point Q(0,0) as our second point. Now we can find slope with 
    P(1,1), and Q(0,0).

    \bigbreak \noindent 
    \begin{large}
        \textbf{If} Slope = $\frac{y2-y1}{x2-x1}$, Then M of PQ $\rightarrow$ $ \frac{1-0}{1-0}$ = 1
    \end{large}

    \bigbreak \noindent 
    \textbf{Lets} get a better approximation by using a point closer to the tangent line
    Lets use Q(0.9, 0.81)

    \bigbreak \noindent 
    \begin{large}
        \textbf{So} M of PQ $\rightarrow$ $\frac{1-0.81}{1-0.9}$ = 1.9
    \end{large}

    \bigbreak \noindent 
    \textbf{Now}, lets get an even closer approximation by using the point Q(0.99, 0.9801)
    
    \bigbreak \noindent 
    \begin{large}
       \textbf{So}, M of  PQ $\rightarrow$ $ \frac{1-0.9801}{1-0.99}$ = 1.99
    \end{large}

    \pagebreak
    \noindent \textbf{Notice}, as the point Q gets closer to P, the slope of PQ is getting closer to 2
    
    \bigbreak \noindent 
    \textbf{We write}, 
    \begin{center}
        \begin{large}
            $\lim\limits_{Q \to P}${M of PQ} = m
        \end{large}
    \end{center}

    \bigbreak \noindent 
    Where \textbf{m} on the right of equation is slope of tangent line at \textbf{P}, 
    And \textbf{M of PQ} is slope of the secent line

    \bigbreak \noindent \bigbreak \noindent 
    \begin{large}
        \textbf{Now,}     
    \end{large}
    \bigbreak \noindent 
    We will use our approximation of $m \approx 2$ to write the equation of the tangent line,
    using the orginial point P(1,1).

    \begin{align*}
        y-1=2\left(x-1\right) \\
        y-1=2x-2 \\
        y=2x-1
    .\end{align*}
    
    
    \pagebreak
    \begin{large}
       \noindent \textbf{The Velocity Problem:} 
    \end{large}
    
    \bigbreak \noindent \bigbreak \noindent 
    \begin{itemize}
        \item Average Velocity: $\frac{distance\ traveled}{time\ elapsed}$, which is 
            represented by the slope of the secent line.
        \item Instantaneous Velocity = Velocity at a given instant of time, which is represented 
            by the slope of the tangent line
    \end{itemize}

    \bigbreak \noindent \bigbreak \noindent  
    \ex{}{If a rock is thrown upward on the planet Mars, with a Velocity of 10 m/s, It's
    height in meters t seconds later is  given by $y=10t-1.86t^2$}

    \bigbreak \noindent 
    \qs{}{Find the average Velocity over the given time intervals:}
    
    \bigbreak \noindent 
    \textbf{(i)} \textbf{[1,2]} $\rightarrow$ 1 and 2 represent values of \textit{t}
    
    \bigbreak \noindent 
    \begin{center}
        Substitute values into equation above
    \end{center}
    \begin{align*}
        y\left(1\right)=10\left(1\right)-1.86\left(1\right)^2 \\
        = 8.14
    .\end{align*}

    \begin{align*}
        y\left(2\right)=10 \left(2\right) - 1.86 \left(2\right) ^2 \\
        = 12.56
    .\end{align*}

    \bigbreak \noindent 
    \textbf{If} Average Velocity = $\frac{distance\ traveled}{time\ elapsed}$ Or better yet
    $ \frac{Change\ in\ height}{change\ in\ time}$

    \bigbreak \noindent 
    \textbf{And} we have the points (1,8.14) and (2,12.56)

    \bigbreak \noindent 
    \textbf{Then,}

    \begin{align*}
        Average\ Velocity = \frac{12.56-8.14}{2-1} \\
        =4.42 m\diagdown s
   .\end{align*}

    \pagebreak
    \textbf{(ii) [1,1.5]}
    
    \begin{center}
        Substitute values into equation above
    \end{center}
    \begin{align*}
        y\left(1\right)=10\left(1\right)-1.86\left(1\right)^2 \\
        = 8.14
    .\end{align*}

    \bigbreak \noindent 
    \begin{align*}
        y \left(1.5\right) = 10 \left(1.5\right) - 1.86 \left(1.5\right) ^2 \\
        =10.815 
    .\end{align*}

    \bigbreak \noindent 
    \textbf{After} solving theses equations we have the points (1,8.14) and (1.5,10.815)
    \bigbreak \noindent 
    \textbf{So,}
    
    \begin{align*}
        Average\ Velocity = \frac{10.815-8.14}{1.5-1} \\
        = 5.35 m \diagdown s
    .\end{align*}

\end{document}
