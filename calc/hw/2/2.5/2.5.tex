\documentclass{report}

\input{~/dev/latex/template/preamble.tex}
\input{~/dev/latex/template/macros.tex}

\title{\Huge{2.5 HW Solutions}}
\author{\huge{Nathan Warner}}
\date{\huge{Jan 31, 2023}}

\begin{document}
    \maketitle
    \begin{Large}
        \noindent \textbf{Question 1:}
    \end{Large} 

    \bigbreak \noindent 
    \textbf{a.)}
    \pf{Solution}{}

    \bigbreak \noindent 
    We can see that the values of a that approach a different number from the right side than
    from the left is values \textbf{\textit{0 and 1}}

    \bigbreak \noindent \bigbreak \noindent \bigbreak \noindent  
    \textbf{b.)} 
    \pf{Solution}{}
    \bigbreak \noindent 
    We can see that the value a decreases without bound at \textbf{\textit{-2}}, Also, -1 is shown to 
    not be defined on the graph, so the answers are \textbf{\textit{-2,-1}}

    \bigbreak \noindent \bigbreak \noindent \bigbreak \noindent 
    \textbf{c.)}
    \pf{Solution}{}
    \bigbreak \noindent 
    We can see that that values of a that are \textbf{\textit{Discontinous}} are \textbf{-2,-1,0,1}.
    We can see that -2 has a hole, -1 is not defined, are zero and 1 have a limit of DNE


    \bigbreak \noindent \bigbreak \noindent \bigbreak \noindent  
    \begin{Large}
        \textbf{Question 2:}
    \end{Large}

    \bigbreak \noindent 
    \textbf{a.)}
    \pf{Solution}{}
    \bigbreak \noindent 
    We can see that $\lim\limits_{x \to a}{f \left(x\right)}$ does not exist at a = 1,5

    \bigbreak \noindent 
    \textbf{b.)}
    \pf{Solution}{}
    \bigbreak \noindent 
    We can see that values \textbf{1,3,5} are Discontinous (not continuous). We can see that 
    with a = 1, the limit is not a finite number, for a = 3, there is a hole in the graph and 
    that $\lim\limits_{x \to a}{f \left(x\right)} \neq f \left(a\right)$. For value a = 5, we can 
    see that the limit does not exist.


    \bigbreak \noindent \bigbreak \noindent \bigbreak \noindent  
    \begin{Large}
        \textbf{Question 3:}
    \end{Large}

    \bigbreak \noindent 
    \textbf{a.)}
    \pf{Solution}{}
    \bigbreak \noindent 
    See hw

    \bigbreak \noindent \bigbreak \noindent \bigbreak \noindent  
    \begin{Large}
        \textbf{Question 4:}
    \end{Large}

    \bigbreak \noindent 
    \textbf{a.)}
    \pf{Solution}{}
    \bigbreak \noindent 
    See hw

    \bigbreak \noindent \bigbreak \noindent \bigbreak \noindent  
    \begin{Large}
        \textbf{Question 5:}
    \end{Large}

    \bigbreak \noindent 
    \textbf{a.)}
    \pf{Solution}{}
    \bigbreak \noindent 
    See hw

    \bigbreak \noindent \bigbreak \noindent \bigbreak \noindent  
    \begin{Large}
        \textbf{Question 6:}
    \end{Large}

    \bigbreak \noindent 
    \textbf{a.)}
    \pf{Solution}{}
    \bigbreak \noindent 
    To state the domain, we must factor the denomonator;

    \begin{align*}
        \left(v+8\right) \left(v-5\right)
    .\end{align*}

    \bigbreak \noindent 
    Now we get that v $\neq$ -8 and v $\neq$ 5. So if we write the domain in interval notation, we get

    \begin{align*}
        \left(-\infty, -8\right) \cup \left(-8,5\right) \cup (5, \infty)
    .\end{align*}

    \bigbreak \noindent \bigbreak \noindent \bigbreak \noindent  
    \begin{Large}
        \textbf{Question 7:}
    \end{Large}

    \bigbreak \noindent 
    \textbf{a.)}
    \pf{Solution}{}
    \bigbreak \noindent 
    See hw


    \bigbreak \noindent \bigbreak \noindent \bigbreak \noindent  
    \begin{Large}
        \textbf{Question 8:}
    \end{Large}

    \bigbreak \noindent 
    \textbf{a.)}
    \pf{Solution}{}
    \bigbreak \noindent 
    Domain of arccos, is [-1,1], so we must find where the arguments are defined, 
    if we set the expression inside the parenthesis $\geq$ - 1, 

    \begin{align*}
        e^t \geq -1 
    .\end{align*}

    \bigbreak \noindent 
    to solve this inequality, we must take the ln of both sides to solve for t 

    \begin{align*}
        t \geq ln(0)
    .\end{align*}

    \bigbreak \noindent 
    But ln(0) is undefined. So we set the value $\geq$ to 1.

    \begin{align*}
        e^2 \geq 1
    .\end{align*}

    \bigbreak \noindent 
    Now if we take the ln of both sides, we get:

    \begin{align*}
        t \geq ln(2)
    .\end{align*}

    \bigbreak \noindent 
    so now we know that the domain of this composite function is:

    \begin{align*}
        (- \infty, ln(2)]
    .\end{align*}


    \bigbreak \noindent \bigbreak \noindent \bigbreak \noindent  
    \begin{Large}
        \textbf{Question 9:}
    \end{Large}

    \bigbreak \noindent 
    \textbf{a.)}
    \pf{Solution}{}
    \bigbreak \noindent 
    \textbf{1.)} 
    \begin{align*}
        f \left(3\right) = 3 \cdot \sqrt{13- \left(3\right)^2} \\ 
        = 6
    .\end{align*}

    \bigbreak \noindent 
    Now we check to see if 6 is within the domain of this function. 

    \bigbreak \noindent 
    Because this is a radical function, we must take the contents inside that radical and set
    $\geq$ 0 

    \begin{align*}
        13-x^2 \geq 0 \\
        -x^2 \geq -13 \\ 
        x^2 \leq 13 
    .\end{align*}

    \bigbreak \noindent 
    So if we take the sqrt of both sides, we get:

    \begin{align*}
        x \leq \sqrt{13}\ \text{and}\ x \leq - \sqrt{13}
    .\end{align*}

    \bigbreak \noindent 
    so our domain is:

    \begin{align*}
        [- \sqrt{13}, \sqrt{13}]
    .\end{align*}

    \bigbreak \noindent 
    Because 3 lies within this domain, we can plug in 3 into f(x) and get the limit:

    \begin{align*}
        \lim\limits_{x \to 3}{3 \cdot \sqrt{13- \left(3\right)^2}} \\
        = 6
    .\end{align*}

    \bigbreak \noindent 
    So 6 is our answer.

    \bigbreak \noindent \bigbreak \noindent \bigbreak \noindent  
    \begin{Large}
        \textbf{Question 10:}
    \end{Large}

    \bigbreak \noindent 
    \textbf{a.)}
    \pf{Solution}{}
    \bigbreak \noindent
    \textbf{See hw}

    \bigbreak \noindent \bigbreak \noindent \bigbreak \noindent  
    \begin{Large}
        \textbf{Question 11:}
    \end{Large}

    \bigbreak \noindent 
    \textbf{a.)}
    \pf{Solution}{}
    \bigbreak \noindent
    \begin{center}
        \begin{Huge}
            \textbf{\href{https://www.wyzant.com/resources/answers/174393/find_the_values_of_a_and_b_that_make_f_continuous_everywhere}{explanation}}
        \end{Huge}
    \end{center}
    
    \end{document}
