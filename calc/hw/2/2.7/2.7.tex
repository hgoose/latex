\documentclass{report}

\input{~/dev/latex/template/preamble.tex}
\input{~/dev/latex/template/macros.tex}

\title{\Huge{2.7 Hw Solutions}}
\author{\huge{Nathan Warner}}
\date{\huge{}}

\begin{document}
    \maketitle
    \begin{Large}
        \noindent \textbf{Question 1:}
    \end{Large}
    \bigbreak \noindent 
    \pf{Solution}{}
    \bigbreak \noindent 
    \textbf{a.)}
    \begin{center}
        If:
    \end{center}
    \begin{align*}
        m_{tan} = \lim\limits_{x \to -3}{ \frac{f(x) - f(a)}{x - a}}
    .\end{align*}
    \begin{center}
        And \textbf{a = -3, f(a) = -18, then:}
    \end{center}
    \begin{align*}
        m_{tan} = \lim\limits_{x \to -3}{ \frac{x^2+9x- \left(-18\right)}{x - \left(-3\right)}} \\ 
        = \lim\limits_{x \to -3}{ \frac{x^2 +9x+18}{x + 3}}
    .\end{align*}
    \begin{center}
        numerator factors into:
    \end{center}
    \begin{align*}
        \lim\limits_{x \to -3}{ \frac{ \left(x+3\right) \left(x+6\right)}{x+3}}
    .\end{align*}
    \begin{center}
        Cancel out common factor:
    \end{center}
    \begin{align*}
        \lim\limits_{x  \to -3}{x+6} 
    .\end{align*}
    \begin{center}
        Plug in -3 for x
    \end{center}
    \begin{align*}
        m_{tan} = -3 + 6 \\ 
        = 3
    .\end{align*}

    \bigbreak \noindent \bigbreak \noindent 
    \textbf{b.)} 
    \bigbreak \noindent 
    \noindent If:
    \begin{align*}
        m_{tan} = \lim\limits_{h \to 0}{ \frac{f(a + h) - f(a)}{h}}
    .\end{align*}
    \bigbreak \noindent And a = -3 and f(a) = -18, and we plug in (a+h) for x:
    \begin{align*}
        m_{tan} = \lim\limits_{h \to 0}{ \frac{ \left(-3 + h\right)^2 + 9 \left(-3 + h\right) - \left(-18\right)}{h}} \\ 
    .\end{align*}
    \bigbreak \noindent 
    And we distribute out the terms:
    \begin{align*}
        m_{tan} = \lim\limits_{h \to 0}{ \frac{h^2-6h+9-27+9h-18}{h}} \\
        = \frac{ h^2 + 3h}{h} \\ 
        = \frac{h \left(h+3\right)}{h} \\ 
        = h + 3
    .\end{align*}
    \bigbreak \noindent 
    Now if we plug in zero: 
    \begin{align*}
        m_{tan} = 0 + 3 \\
        = 3
    .\end{align*}

    \bigbreak \noindent \bigbreak \noindent 
    \textbf{c.)} 
    The equation of the tangent line if:
    \begin{align*}
        y - y_1 = m \left(x - x_1\right) \\ 
    .\end{align*}
    \bigbreak \noindent 
    Then:
    \begin{align*}
         y - \left(-18\right) = 3 \left(x - \left(-3\right)\right) \\
         y + 18 = 3 \left(x+3\right) \\ 
         y + 18 = 3x + 9 \\ 
         y = 3x - 9
    .\end{align*}


    \bigbreak \noindent \bigbreak \noindent \bigbreak \noindent 
    \begin{Large}
        \textbf{Question 2:}
    \end{Large}
    \bigbreak \noindent 
    \pf{Solution}{}
    \bigbreak \noindent 
    We know that:
    \begin{align*}
        m_{tan} = \lim\limits_{x \to a}{\frac{f(x) - f(a)}{x -a}}
    .\end{align*}
    \bigbreak \noindent 
    And a = 6 and f(a) = 19:
    \begin{align*}
        \lim\limits_{x \to 6}{ \frac{2x^2-9x-1-19}{x-19}} \\
        = \lim\limits_{x \to 6}{ \frac{2x^2 -9x -18}{x - 19}}
    .\end{align*}
    \bigbreak \noindent 
    factor using the x method:
    \begin{align*}
        \lim\limits_{x \to 6}{ \frac{ \left(2x+3\right) \left(x-6\right)}{x-6}} \\ 
        = \lim\limits_{x \to 6}{2x+3} \\ 
        = 2 \left(6\right) + 3 \\ 
        =15
    .\end{align*}
    \bigbreak \noindent 
    Plug $m_{tan} = 15$ into \textbf{\textit{Point slope form equation}} to get equation of tangent line:

    \begin{align*}
        y - 19 = 15 \left(x-6\right) \\ 
        y - 19 = 15x - 90 \\ 
        y = 15x - 71
    .\end{align*}

    \bigbreak \noindent \bigbreak \noindent \bigbreak \noindent 
    \begin{Large}
        \textbf{Question 3:}
    \end{Large}
    \bigbreak \noindent 
    \pf{Solution}{}
    \bigbreak \noindent 
    \textbf{a.)}
    \bigbreak \noindent If:
    \begin{align*}
        m_{tan} = \lim\limits_{h \to 0}{ \frac{f(a+h) - f(a)}{h}}
    .\end{align*}
    \bigbreak \noindent Then:
    \begin{align*}
        m_{tan} = \lim\limits_{h \to 0}{ \frac{4 + 5\left(a+h\right)^2 - 2 \left(a+h\right)^3 - \left(4+5a^2-2a^3\right)}{h}}
    .\end{align*}
    \bigbreak \noindent     
    \bigbreak \noindent 
    Distribute -1 to each term in $4+5a^2-2a^3$
    \begin{align*}
        = -4 -5a^2+2a^3
    .\end{align*}
    \bigbreak \noindent 
    Foil out $-2 \left(a+h\right)^3$
    \begin{align*}
        = -2a^3-2h^3-6a^2h-6ah^2
    .\end{align*}
    \bigbreak \noindent 
    Foil out $5 \left(a+h\right)^2$
    \begin{align*}
        = 5a^2+5h^2+10ah
    .\end{align*}
    \bigbreak \noindent 
    And we also have the 4 in the beginning, so combine like terms

    \begin{align*}
        -2h^3-6a^2h-6ah^2+5h^2+10ah
    .\end{align*}
    \bigbreak \noindent 
    Add to equation:
    \begin{align*}
        \lim\limits_{h \to 0}{ \frac{-2h^3-6a^2h-6ah^2+5h^2+10ah}{h}}
    .\end{align*}
    \bigbreak \noindent 
    factor out a \textbf{\textit{h}}, and cancel out common term \textbf{\textit{h}}
    \begin{align*}
        \lim\limits_{h \to 0}{ \frac{h \left(-2h^2-6a^2-6ah+5h+10a\right)}{h}} \\
        =-2h^2-6a^2-6ah+5h+10a 
    .\end{align*}
    \bigbreak \noindent 
    Plug in zero for each h
    \begin{align*}
        -2 \left(0\right)^2-6a^2-6a \left(0\right) + 5 \left(0\right)+10a \\
        = -6a^2+10a
    .\end{align*}
    \bigbreak \noindent \bigbreak \noindent 
    \textbf{b.)}
    Plug in 1 for x,
    \begin{align*}
        m = -6 \left(1\right)^2 + 10 \left(1\right) \\ 
        = 4
    .\end{align*}
    \bigbreak \noindent 
    Plug into point slope form equation 
    \begin{align*}
        y - 7 = 4 \left(x - 1\right) \\ 
        y = 4x+3
    .\end{align*}

    \bigbreak \noindent \bigbreak \noindent \bigbreak \noindent 
    \begin{Large}
       \textbf{Question 4:}
    \end{Large}
    \bigbreak \noindent 
    \pf{Solution}{}
    \bigbreak \noindent 
    \textbf{Part b.)} 
    \begin{align*}
        16t^2 = 36 \\ 
         t^2 = \frac{36}{16} \\ 
         t = \frac{ \sqrt{36}}{ \sqrt{16}} \\
         t = \frac{6}{4} \\ 
         t = \frac{3}{2} \\
         t = 1.5 s 
    .\end{align*}
    \bigbreak \noindent 
    \textbf{Part d.)}
    \begin{align*}
        \lim\limits_{h \to 0}{ \frac{16 (a+h)^2 - 16a^2}{h}} \\ 
        \lim\limits_{h \to 0}{ \frac{16a^2+16h^2+36ah-16a^2}{h}} \\
        \lim\limits_{h \to 0}{ \frac{16h^2+36ah}{h}} \\
        \lim\limits_{h \to 0}{ \frac{h (16h+36a)}{h}} \\ 
        \lim\limits_{h \to 0}{16h+36a} \\
        16(0) + 36(1.5) \\
        = 48
    .\end{align*}

    \bigbreak \noindent \bigbreak \noindent \bigbreak \noindent 
    \begin{Large}
        \textbf{Question 5:}
    \end{Large}
    \bigbreak \noindent 
    \pf{Solution}{}
    \bigbreak \noindent 

    \bigbreak \noindent \bigbreak \noindent \bigbreak \noindent 
    \begin{Large}
        \textbf{Question 6:}
    \end{Large}
    \bigbreak \noindent 
    \pf{Solution}{}
    \bigbreak \noindent 

    \bigbreak \noindent \bigbreak \noindent \bigbreak \noindent 
    \begin{Large}
        \textbf{Question 7:}
    \end{Large}
    \bigbreak \noindent 
    \pf{Solution}{}
    \bigbreak \noindent 

    \bigbreak \noindent \bigbreak \noindent \bigbreak \noindent 
    \begin{Large}
        \textbf{Question 8:}
    \end{Large}
    \bigbreak \noindent 
    \pf{Solution}{}
    \bigbreak \noindent 

    \bigbreak \noindent \bigbreak \noindent \bigbreak \noindent 
    \begin{Large}
        \textbf{Question 9:}
    \end{Large}
    \bigbreak \noindent 
    \pf{Solution}{}
    \bigbreak \noindent 

    \bigbreak \noindent \bigbreak \noindent \bigbreak \noindent 
    \begin{Large}
        \textbf{Question 10:}
    \end{Large}
    \bigbreak \noindent 
    \pf{Solution}{}
    \bigbreak \noindent 
    

\end{document}
