\documentclass{report}

\input{~/dev/latex/template/preamble.tex}
\input{~/dev/latex/template/macros.tex}

\title{\Huge{Chapter 10 Review and Discuss}}
\author{\huge{Nathan Warner}}
\date{\huge{March 12, 2023}}
\pagestyle{fancy}
\fancyhf{}
\rhead{REVIEW AND DISCUSS}
\lhead{\leftmark}
\cfoot{\thepage}
% \usepackage[default]{sourcecodepro}
% \usepackage[T1]{fontenc}

\pgfpagesdeclarelayout{boxed}
{
  \edef\pgfpageoptionborder{0pt}
}
{
  \pgfpagesphysicalpageoptions
  {%
    logical pages=1,%
  }
  \pgfpageslogicalpageoptions{1}
  {
    border code=\pgfsetlinewidth{1.5pt}\pgfstroke,%
    border shrink=\pgfpageoptionborder,%
    resized width=.95\pgfphysicalwidth,%
    resized height=.95\pgfphysicalheight,%
    center=\pgfpoint{.5\pgfphysicalwidth}{.5\pgfphysicalheight}%
  }%
}

\pgfpagesuselayout{boxed}

\begin{document}
    \maketitle
    \qs{}{Analyze how similarity and contrast are both needed in a design. 
}
    \bigbreak \noindent 
    \pf{Answer}{}
    \bigbreak \noindent 
    In a design, similarity and contrast play a critical role in capturing and retaining the viewer's attention.
    While similarity creates a sense of harmony and unity, contrast brings in variation and adds interest.
    By balancing these two elements, designers can create a visually appealing and engaging design that effectively communicates the intended message.

    \bigbreak \noindent \bigbreak \noindent 
    \qs{}{Describe what makes balance important in a design. 
}
    \bigbreak \noindent 
    \pf{Answer}{}
    \bigbreak \noindent 
    Balance is a crucial aspect of design that ensures that all the visual elements on a page are appropriately distributed, creating a sense of harmony and equilibrium.
    Achieving balance in a design helps to prevent any single element from overpowering the others,
    It can be achieved through symmetrical, asymmetrical, or radial balance.

    \bigbreak \noindent \bigbreak \noindent 
    \qs{}{Analyze the effect an out-of-alignment element on a page has on the human brain and what this can do to the design being accepted.
}
    \bigbreak \noindent 
    \pf{Answer}{}
    \bigbreak \noindent 
    An out-of-alignment element on a page can have a significant impact on the viewer's perception of a design.
    This can cause cognitive dissonance and make the design less appealing or even unbalanced.
    By carefully aligning all the visual elements, designers can create a cohesive and balanced design that is more likely to be accepted by the viewer.
    
    
    \bigbreak \noindent \bigbreak \noindent 
    \qs{}{Compare and contrast the three types of balance in a design. 
}
    \bigbreak \noindent 
    \pf{Answer}{}
    \bigbreak \noindent 
    There are three types of balance in a design: symmetrical, asymmetrical, and radial.
    Symmetrical balance involves creating a mirrored image on either side of a central axis.
    Asymmetrical balance involves distributing visual elements unequally across a page while still maintaining balance.
    Radial balance involves placing visual elements in a circular or spiral pattern around a central point.
    
    \bigbreak \noindent \bigbreak \noindent 
    \qs{}{Analyze the effects of bleeding a graphic on a page. 
}
    \bigbreak \noindent 
    \pf{Answer}{}
    \bigbreak \noindent 
    Bleeding a graphic on a page can have both positive and negative effects on the overall design.
    On the one hand, it can help to create a sense of continuity and flow across multiple pages.
    On the other hand, it can also lead to a cluttered and confusing design if not used effectively.
    
    \bigbreak \noindent \bigbreak \noindent 
    \qs{}{Compare and contrast master pages and templates. 
}
    \bigbreak \noindent 
    \pf{Answer}{}
    \bigbreak \noindent 
    Master pages and templates are both useful tools for creating consistent designs, but they differ in their scope and function.
    Master pages provide a framework for creating consistent page layouts, while templates provide a framework for creating consistent designs across multiple pages.
    While master pages are typically used for creating page layouts in publications like magazines and books, templates are more commonly used in digital design applications like websites and mobile apps.
    
    \bigbreak \noindent \bigbreak \noindent 
    \qs{}{Evaluate the uses of PDF file format. What information is useful to obtain from a printer to avoid problems at the press and what is an alternative if that information is not available? 
}
    \bigbreak \noindent 
    \pf{Answer}{}
    \bigbreak \noindent 
    The PDF file format is widely used in the printing industry because it provides a high level of compatibility and security.
    When working with a printer, it is important to obtain information about the printer's specific requirements and capabilities to avoid any problems during the printing process.
    If this information is not available, alternative file formats like EPS or TIFF may be used instead.
    
    \bigbreak \noindent \bigbreak \noindent 
    \qs{}{Evaluate the use of a grid in creating a design. 
}
    \bigbreak \noindent 
    \pf{Answer}{}
    \bigbreak \noindent 
    The use of a grid in creating a design helps to provide structure and organization while also ensuring that all the visual elements on a page are appropriately sized and spaced.
    Grids can be used to create a wide variety of designs, from simple layouts to complex compositions.
    By following the guidelines provided by a grid, designers can create designs that are both visually appealing and easy to read.
    
    \bigbreak \noindent \bigbreak \noindent 
    \qs{}{Analyze the importance of each of the three forms of rhythm in a design and how it relates to music. 
}
    \bigbreak \noindent 
    \pf{Answer}{}
    \bigbreak \noindent 
    There are three forms of rhythm in a design: regular, flowing, and progressive.
    Regular rhythm involves creating a consistent pattern of visual elements.
    Flowing rhythm involves creating a sense of movement and fluidity across a page.
    Progressive rhythm involves creating a sense of progression or forward movement through the use of visual elements.
    Each of these forms of rhythm is important in creating a visually engaging design that captures and retains the viewer's attention, much like music.
    The use of white space in a publication is important for creating a sense of balance and clarity on a page.
    
    \bigbreak \noindent \bigbreak \noindent 
    \qs{}{Defend the use of white space in a publication with an evaluation of some of the pitfalls of using white space.}
    \bigbreak \noindent 
    \pf{Answer}{}
    \bigbreak \noindent 
    White space (also known as negative space) is an important element in design that can help to create balance and clarity.
    White space allows readers to focus on the content and makes it easier to read and understand.
    White space can also create a sense of elegance and sophistication in a design, especially when used in conjunction with high-quality typography.
    
    
    
\end{document}
