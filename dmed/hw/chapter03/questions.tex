\documentclass{report}

\input{~/dev/latex/template/preamble.tex}
\input{~/dev/latex/template/macros.tex}

\title{\Huge{Chapter 3 Review and Discuss Questions}}
\author{\huge{Nathan Warner}}
\date{\huge{Jan 27, 2023}}

\begin{document}
    \maketitle
    \qs{}{Name the two types of digital imaging programs and differentiate them. 
} 
    \pf{Answer}{}
    \noindent The two types of digital imaging programs are:
    \begin{itemize}
        \item Paint Programs
        \item Drawing Programs
    \end{itemize}
    \noindent Paint Programs use pixels to create images whereas drawing programs use vectors
    or lines that are defined by math. Vector images can be resized without distorting the image
    \bigbreak \noindent 

    \qs{}{Identify the advantage of using a raster graphic. 
} 
    \pf{Answer}{}
    \noindent Raster images carry a wide range of color options for the pixels that make
    up the image.
    \bigbreak \noindent 
    \qs{}{identify the advantage of using a vector graphic. 
} 
    \pf{Answer}{}
    \noindent Vector Graphics can be resized without distorting the image.
    \bigbreak \noindent 
    \qs{}{Describe the advantages and disadvantages of cloud computing. 
} 
    \pf{Answer}{}
    \noindent Cloud Computing can give users access to imaging software through the interent. 
    These services are usually pretty cheap. However you must have an internet connection to use them.
    \bigbreak \noindent 
    \qs{}{Explain why it is important to understand file extensions. 
} 
    \pf{Answer}{}
    \noindent It is important to understand file extensions because some programs can only
    import images with certain formats. Also, some images will look better if saved in a 
    certain format rather than a different one. Furthermore, sometimes you need a smaller image file
    to be loaded quickly on the internet. 
    \bigbreak \noindent 
    \qs{}{List at least four types of bitmap image extensions. 
} 
    \pf{Answer}{}
    \begin{itemize}
        \item BMP
        \item PNG
        \item JFG
        \item TIF
    \end{itemize}
    \bigbreak \noindent 
    \qs{}{Differentiate between an AI vector file and an SVG or EPS one. 
} 
    \pf{Answer}{}
    \noindent An AI vector file is a native vector file format and comes from 
    Adobe Illustrator. SVG and EPS vector formats are nonnative. 
    \bigbreak \noindent 
    \qs{}{Explain why images are compressed. 
} 
    \pf{Answer}{}
    \noindent Images are compressed in order to reduce the size that the image takes up on disk.
    \bigbreak \noindent 
    \qs{}{Differentiate between lossy and lossless compression. 
} 
    \pf{Answer}{}
    \noindent Lossy compression deletes or changes some pixels while lossless reduces the file size
    without losing any pixels.
    \bigbreak \noindent 
    \qs{}{Explain what happens to an image if you resize it with resampling selected. 
} 
    \pf{Answer}{}
    \noindent Resampling allows for adding or removing pixels when you resize an image. 
    When resampling is on, the resolution of the file will not change if you change the 
    size of the image
\end{document}
