\documentclass{report}

\input{~/dev/latex/template/preamble.tex}
\input{~/dev/latex/template/macros.tex}

\title{\Huge{Chapter 4 Review and Discuss}}
\author{\huge{Nathan Warner}}
\date{\huge{}}

\pgfpagesdeclarelayout{boxed}
{
  \edef\pgfpageoptionborder{0pt}
}
{
  \pgfpagesphysicalpageoptions
  {%
    logical pages=1,%
  }
  \pgfpageslogicalpageoptions{1}
  {
    border code=\pgfsetlinewidth{1.5pt}\pgfstroke,%
    border shrink=\pgfpageoptionborder,%
    resized width=.95\pgfphysicalwidth,%
    resized height=.95\pgfphysicalheight,%
    center=\pgfpoint{.5\pgfphysicalwidth}{.5\pgfphysicalheight}%
  }%
}

\pgfpagesuselayout{boxed}

\begin{document}
    \maketitle

    \bigbreak \noindent \bigbreak \noindent 
    \qs{}{Identify what has replaced film in storing images on cameras and state the advantage of this new storage medium. 
}
    \bigbreak \noindent 
    \pf{Answer}{Film has been superseded by digital storage options including memory cards and internal storage for photos taken with cameras. One benefit of this is the convenience of having unlimited storage space and the ability to rapidly examine and remove photographs.}
    \bigbreak \noindent \bigbreak \noindent 

    \bigbreak \noindent \bigbreak \noindent 
    \qs{}{Explain how digital cameras increase your chance of taking the perfect picture.
    }
    \bigbreak \noindent 
    \pf{Answer}{Digital cameras' automatic settings, rapid reviews, and the flexibility to change parameters like exposure and white balance boost the likelihood of capturing the ideal shot. This enables the photographer to make adjustments in real-time and ultimately raises the likelihood that the desired image will be captured.}
    \bigbreak \noindent \bigbreak \noindent 

    \bigbreak \noindent \bigbreak \noindent 
    \qs{}{Explain the difference between an SLR camera and a point-and-shoot camera. 
    }
    \bigbreak \noindent 
    \pf{Answer}{An SLR (single-lens reflex) camera has a mirror that reflects the light passing through the lens, allowing the photographer to view the subject through the same lens that takes the picture. A point-and-shoot camera has a compact design and is simpler to use, with most settings being automatic.
    }
    \bigbreak \noindent \bigbreak \noindent 

    \bigbreak \noindent \bigbreak \noindent 
    \qs{}{Differentiate between digital and optical zoom. 
    }
    \bigbreak \noindent 
    \pf{Answer}{Digital zoom enlarges an image by cropping and resizing the picture, which can lead to a loss of image quality. Optical zoom uses the lens to physically move and magnify the subject, resulting in a clearer image.
    }
    \bigbreak \noindent \bigbreak \noindent 

    \bigbreak \noindent \bigbreak \noindent 
    \qs{}{Describe why a photograph that will be used for a web page would be saved as a JPG with a resolution of 640 × 480. 
}
    \bigbreak \noindent 
    \pf{Answer}{A photo saved as JPG format with a resolution of 640 × 480 is suitable for web use because the format is efficient in terms of file size, while the resolution is small enough to load quickly on a web page.
}
    \bigbreak \noindent \bigbreak \noindent 

    \bigbreak \noindent \bigbreak \noindent 
    \qs{}{Explain the advantages in setting a camera so it saves an image as a raw file type. 
}
    \bigbreak \noindent 
    \pf{Answer}{A raw file type provides greater control over the editing process, as it retains all of the data captured by the camera's sensor.}
    \bigbreak \noindent \bigbreak \noindent 

    \bigbreak \noindent \bigbreak \noindent 
    \qs{}{List three rules for keeping your camera in good working order and three procedures for maintaining safety while working with a digital camera. 
}
    \bigbreak \noindent 
    \pf{Answer}{}
    \begin{itemize}
        \item Keep the camera and lens clean and free of dust.
        \item Store the camera in a dry, cool place when not in use.
        \item Handle the camera with care and avoid exposing it to extreme temperatures and moisture.
    \end{itemize}
    \bigbreak \noindent \bigbreak \noindent 

    \bigbreak \noindent \bigbreak \noindent 
    \qs{}{Describe what composition has to do with taking a good photograph. 
}
    \bigbreak \noindent 
    \pf{Answer}{The arrangement of elements within a photograph is referred to as composition, and it significantly affects the image's overall impression. A more engaging and visually beautiful image results from effective composition, which helps direct the viewer's eye.}
    \bigbreak \noindent \bigbreak \noindent 

    \bigbreak \noindent \bigbreak \noindent 
    \qs{}{Define focal point and explain how this can be changed in an image.  
}
    \bigbreak \noindent 
    \pf{Answer}{The area of the image that is sharpest in focus and draws the viewer's eye is referred to as the focal point.
By modifying the camera's aperture, which governs the depth of field, and focus point, which establishes which area of the image is sharply in focus, the focal point of an image can be adjusted.
A shallow depth of field can be created by using a wider aperture, making the background blurry and allowing the subject to stand out from it, while a deeper depth of field and a sharper background are produced by using a smaller aperture. The focal point can be moved to a new area of the scene by focusing on a different area of the image.}
    \bigbreak \noindent \bigbreak \noindent 

    \bigbreak \noindent \bigbreak \noindent 
    \qs{}{State why the rule of thirds is important when taking a photograph.  
}
    \bigbreak \noindent 
    \pf{Answer}{Because it aids in establishing balance and interest in an image, the rule of thirds is crucial in photography.
According to the rule, the topic or important parts should be placed along these lines or at their intersections once the frame has been divided into three portions both horizontally and vertically.
By adhering to the rule of thirds, the picture appears more visually pleasing and balanced.}
    \bigbreak \noindent \bigbreak \noindent 

    \bigbreak \noindent \bigbreak \noindent 
    \qs{}{Assess what makes a background that frames the subject effective in an image. 
}
    \bigbreak \noindent 
    \pf{Answer}{When framing a subject in an image, the background should be visually pleasing and complimentary to the subject without competing with or distracting from it.
In order to help direct the viewer's attention to the subject, an effective background should be distinct from the subject.}
    \bigbreak \noindent \bigbreak \noindent 

    \bigbreak \noindent \bigbreak \noindent 
    \qs{}{Cite how the use of lines within a photograph can help to unify elements in an image.
}
    \bigbreak \noindent 
    \pf{Answer}{The use of lines in a photograph can serve to tie the various aspects together and give the picture a sense of movement and direction.
By connecting various aspects and drawing the viewer's attention to the subject or particular areas of the image, lines can guide the viewer's eye through the image.}
    \bigbreak \noindent \bigbreak \noindent 

    \bigbreak \noindent \bigbreak \noindent 
    \qs{}{Consider how the ability to catch action shots has changed photography. 
}
    \bigbreak \noindent 
    \pf{Answer}{With the development of digital photography technology, it has become much easier to take action photos.
Photographers can freeze fast-moving action and catch split-second moments with digital cameras' high shutter speeds and continuous shooting capabilities that wouldn't have been possible with conventional film cameras.}
    \bigbreak \noindent \bigbreak \noindent 

    \bigbreak \noindent \bigbreak \noindent 
    \qs{}{Explain briefly the three components that a camera uses to capture an image. 
}
    \bigbreak \noindent 
    \pf{Answer}{The three components a camera uses to capture an image are the lens, the image sensor, and the shutter.
The lens focuses light onto the image sensor, which converts the light into an electrical signal. The shutter regulates the amount of time that light is allowed to reach the image sensor.}
    \bigbreak \noindent \bigbreak \noindent 


    \bigbreak \noindent \bigbreak \noindent 
    \qs{}{Differentiate between a photograph using a low ISO setting versus one with a high ISO setting. }
    \bigbreak \noindent 
    \pf{Answer}{Low ISO photography will result in a cleaner, less noisy image because it is less sensitive to light.
A high ISO setting can be utilized in low-light situations to boost the camera's sensitivity to light and provide a brighter image.
}
    \bigbreak \noindent \bigbreak \noindent 

    \bigbreak \noindent \bigbreak \noindent 
    \qs{}{Describe white balance on a camera setting.}
    \bigbreak \noindent 
    \pf{Answer}{White balance is a setting in a digital camera that adjusts the color temperature of an image to produce accurate colors.
Without proper white balance, images can appear too warm or too cool, with a yellow or blue tint.}
    \bigbreak \noindent \bigbreak \noindent 

    \bigbreak \noindent \bigbreak \noindent 
    \qs{}{Define an f-number. }
    \bigbreak \noindent 
    \pf{Answer}{The f-stop, commonly referred to as the f-number, is a numerical figure that represents the aperture size in a camera lens.
The ratio of the lens's focal length to its aperture's width is represented by the f-number.}
    \bigbreak \noindent \bigbreak \noindent 

    \bigbreak \noindent \bigbreak \noindent 
    \qs{}{Define a megapixel and explain why it is important for a camera.}
    \bigbreak \noindent 
    \pf{Answer}{A megapixel, or million pixels, is a unit of measurement for the resolution of digital images. The amount of detail that can be caught in an image and the degree of enlargement that may be achieved are both impacted by a camera's megapixel count.}
    \bigbreak \noindent \bigbreak \noindent 

    
\end{document}
