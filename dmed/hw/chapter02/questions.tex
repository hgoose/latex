\documentclass{report}

\input{~/dev/latex/template/preamble.tex}
\input{~/dev/latex/template/macros.tex}

\title{\Huge{Chapter 02 Review and Discuss}}
\author{\huge{Nathan Warner}}
\date{\huge{Jan 20, 2023}}

\begin{document}
    \maketitle
    \section{\Large{Review and Discuss Questions (Chapter 2)}}
    \line(1,0){470}
    \bigbreak \noindent 

    \qs{}{Compare and contrast physical property versus intellectual property.}
    \pf{Answer}{}
    Physical property refers to tangible assets that have a physical existence such as 
    land, buildings, equipment, and natural resources. These properties can be bought, 
    sold, and owned, and their value is typically determined by the market.

    Intellectual property, on the other hand, refers to intangible assets such as ideas, 
    inventions, and creative works that are the product of human intellect. This can include 
    legal forms of protection such as patents, trademarks, copyrights, and trade secrets. The 
    rights to these properties are granted by the government to protect the creators or owners 
    of the intellectual property and to enable them to control its use and distribution. The 
    value of intellectual property is often determined by its potential commercial value, 
    as well as its ability to generate revenue through licensing or sales.
    \bigbreak \noindent 

    \qs{}{List at least five examples of intellectual property that can be assigned copyright protection. }
    \pf{Answer}{}
    \begin{enumerate}
        \item Literary works
        \item Musical works, including any accompanying words
        \item Dramatic works, including any accompanying music
        \item Pantomimes and choreographic works
        \item Pictorial, graphic, and sculptural works
    \end{enumerate}
    
    \bigbreak \noindent 

    \qs{}{Explain why it is important to register a copyright.}
    \pf{Answer}{}
    \begin{enumerate}
        \item First, registering a copyright serves as a reminder that the work is protected under the law.
        \item Second, it clearly establishes the date when the protection for the work begins, which can be used as evidence in case of copyright infringement.
        \item Third, it makes it easier for someone to locate the copyright owner and seek permission to use the work as the owner's name and contact information is recorded in the registration. 
    \end{enumerate}
    
    \bigbreak \noindent 

    \qs{}{Describe the two symbols used to indicate a trademark and explain how they differ.}
    \pf{Answer}{}
    There are two symbols used to indicate trademark status of a word, phrase, 
    or symbol, these are "TM" and the "R" in a circle. "TM" is used to indicate that a word, 
    phrase, or symbol is being claimed as a trademark, whether it is registered or not. The "R" 
    in a circle indicates that the trademark is registered with the government agency responsible for trdemarks. 
    The main difference between these two markings is that the "R" in a circle offers more legal protection 
    and can be used as evidence of registrtion in court, while the "TM" symbol is just a claim of trademark use.
    \bigbreak \noindent 

    \qs{}{Indicate who may claim the right of fair use. }
    \pf{Answer}{}
    The right of fair use can be claimed by anyone, including individuals, 
    organizations, and businesses. Fair use is a legal doctrine that allows the 
    use of copyrighted material without permission in certain circumstances, such as 
    for the purposes of criticism, commentary, news reporting, teaching, scholarship, 
    or research.    
    \bigbreak \noindent 

    \qs{}{Compare and contrast the APA and MLA styles of citing a website.}
    \pf{Answer}{}
    \noindent \textbf{APA website citation}
    \bigbreak \noindent 
    "Peterson, E. (1993, March 15) Weather prediction as  \\
    an unstudied art. Retrieved from \\ 
    http://www.weatherscience.org/peterson/article.html"

    \bigbreak \noindent 
    \textbf{MLA website citation}
    \bigbreak \noindent 
    "Peterson, Emil. “Weather Prediction as an Unstudied Art.” \\ 
    Weather Science. 15 Mar. 1993. Web. 22 May 2017. \\ 
    http://www.weatherscience.org/peterson/article.html"

    \bigbreak
    As you can see by these examples, They both declare the source, and the title in the 
    first part of the citation, however, you can see that the position of the dates differ, 
    for example, the position of the date in the APA citation comes directly after the name of the source, 
    whereas in the MLA citation, the data comes directly after the name of the title. Furthermore, 
    the APA citation reads "Retrieved from" before the url is inserted. 
    \bigbreak \noindent 

    \qs{}{Differentiate between plagiarism and copyright infringement.}
    \pf{Answer}{}
    Plagiarism is when someone uses someone else's work and presents it as their own. 
    Copyright infringement is when someone uses someone else's work without permission.
    \bigbreak \noindent 

    \qs{}{Differentiate between plagiarism and piracy. }
    \pf{Answer}{}
    \textbf{Piracy} is the unauthorized use or reproduction of another's work, it's a form of copyright infringement.
    \textbf{Plagiarism} is the act of using someone else's work and presenting it as your own, 
    without giving credit to the original author. It is a form of academic dishonesty.
    \bigbreak \noindent 

    \qs{}{Explain why ethics related to technology are difficult to determine.}
    \pf{Answer}{}
    Ethics related to technology are difficult to determine because technology is 
    constantly evolving, and what may be considered ethical today may not be in the 
    future. Additionally, technology often operates in gray areas, where the line 
    between right and wrong is not clear, and different perspectives may lead to different 
    conclusions about what is ethical.
    \bigbreak \noindent 

    \qs{}{Summarize why file sharing is an ethical problem.}
    \pf{Answer}{}
    File sharing is an ethical problem because it involves the unauthorized 
    sharing of copyrighted material, which is illegal and can harm the rights 
    and financial compensation of the copyright holder. It also raises questions 
    about the moral implications of sharing and accessing someone else's work without 
    permission.
    \bigbreak \noindent 


\end{document}
