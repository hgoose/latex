\documentclass{report}

\input{~/dev/latex/template/preamble.tex}
\input{~/dev/latex/template/macros.tex}

\title{\Huge{Chapter 5 Notes}}
\author{\huge{Nathan Warner}}
\date{\huge{Feb 7, 2023}}

\pgfpagesdeclarelayout{boxed}
{
  \edef\pgfpageoptionborder{0pt}
}
{
  \pgfpagesphysicalpageoptions
  {%
    logical pages=1,%
  }
  \pgfpageslogicalpageoptions{1}
  {
    border code=\pgfsetlinewidth{1.5pt}\pgfstroke,%
    border shrink=\pgfpageoptionborder,%
    resized width=.95\pgfphysicalwidth,%
    resized height=.95\pgfphysicalheight,%
    center=\pgfpoint{.5\pgfphysicalwidth}{.5\pgfphysicalheight}%
  }%
}

\pgfpagesuselayout{boxed}
\begin{document}
    \maketitle

    \bigbreak \noindent 
     \begin{Large}
        \noindent \textbf{Learning Outcomes:}
     \end{Large}
     \bigbreak \noindent \bigbreak \noindent
     \begin{enumerate}
            \item Understand the difference between RGB, CMYK, and HSB color models.
            \item  Specify the appropriate color mode based on output.
            \item  Become familiar with basic color terminology.
           \item  Identify and use basic color theory to choose colors.
           \item  Recognize the advantages of color matching systems
     \end{enumerate}
  \bigbreak \noindent \bigbreak \noindent \bigbreak
    \begin{Large}
        \noindent \textbf{Key Terms:}
     \end{Large}
     
     \bigbreak \noindent \bigbreak \noindent
    \begin{itemize}
        \item \textbf{additive color mixing:} Combining three different colors of light at different intensities to produce a whole range of other colors
        \item \textbf{analogous colors:} Colors that are next to one another on a color wheel
        \item \textbf{brightness:} In the HSB color model, the measure of how light or dark a color is on a scale of 0 to 100\%
        \item \textbf{CMYK color model:} A color model based on cyan, magenta, yellow, and black pigments used to create full color on printed materials
        \item \textbf{color harmony:} A cohesive and pleasing combination created by a group of colors
        \item \textbf{color mode:} A way to indicate to a computer what color model to use when representing colors
        \item \textbf{color model:} A group of colors identified in a way that computers can understand
        \item \textbf{color theme:} A combina-tion of different hues that work together to create color harmony
        \item \textbf{color theory:} A set of guidelines about how colors communicate feelings and how they combine to create the best look and feel for a project
        \item \textbf{color wheel:} A visual  representation of primary, secondary, and tertiary colors that can be useful for understanding and using  basic concepts in color theory
        \item \textbf{complementary colors:} Colors that are opposite one another on a color wheel
        \item \textbf{gamut:} The range of colors that can be produced by the primary colors in a particular color model
        \item \textbf{grayscale:} A range of grays from white to black with all the variations in between
        \item \textbf{HSB color model:} A color model based on human perception of color that uses hue, saturation, and bright-ness to define a color
        \item \textbf{hue:} In the HSB color model, the general color expressed by a value between 0 and 360 degrees,In color theory, another word for color.
        \item \textbf{neutral color:} In color theory, black, white, and gray used to change the nature of hues, creating shades, tints, and tones
        \item \textbf{out of gamut} Refers to colors that are part of the range of one color model but not another
        \item \textbf{Pantone Matching System (PMS):} A standard set of colors and associated inks that make it easy to repro-duce a color in printed mate-rial consistently
        \item \textbf{primary color:} A basic color that cannot be created by mixing other colors
        \item \textbf{RGB color model:} A color model that uses red, green, and blue primary colors plus different intensities of light to create colors on an electronic display like a com-puter screen
        \item \textbf{RGB triplet:} The combina-tion of numbers indicating light intensity for the red, green, and blue primary col-ors in the RGB color model and representing a certain color within the model. Can be considered the “name” of a color in the RGB color model
        \item \textbf{saturation} In the HSB color model, the intensity of a hue on a scale from 0 to 100\%
        \item \textbf{secondary color:} A color created when two primary colors are mixed
        \item \textbf{shade:} A hue mixed with black
        \item \textbf{spot colors:} Colors that are generated by a single ink, rather than created by blending ink colors
        \item \textbf{subtractive color mixing:} Mixing primary pigment colors to absorb different amounts of light and create a range of colors
        \item \textbf{tertiary color:} The color created when a primary and a secondary color are mixed
        \item \textbf{tint:} A hue mixed with white
        \item \textbf{tone:} A hue mixed with gray
        \item \textbf{Web-safe colors:} 216 colors that all users can see, regardless of their computer displays
         \end{itemize}

         \bigbreak \noindent \bigbreak \noindent
         \begin{Large}
             \noindent \textbf{Key Concepts:}
         \end{Large}
         \bigbreak \noindent
        \begin{itemize}
            \item The RGB color model uses red, green, and blue and different intensities of light to create a range of colors. It’s used to show color on an electronic display. 
            \item The CMYK color model uses cyan, magenta, yellow, and black pigments to create a range of colors. CMYK is used to reproduce colors on printed materials. 
            \item HSB is a color model based on human perception of color (hue, saturation, and brightness). It is used to make adjusting colors easier and can be used in either RGB or CMYK mode. 
            \item The color mode tells the computer which color model data to use to represent  colors.
            \item The two main color modes are RGB and CMYK. You use RGB mode for projects that will be viewed on a screen and CMYK mode for files that will be printed. 
            \item Using a color wheel and understanding concepts such as complementary and analogous colors as well as terms such as shade, tint, and tone can help you recognize and use color theory in your projects. 
            \item You can use different colors to set different moods in your projects. 
            \item Using color themes such as monochromatic, complementary, analogous, and triadic can create color harmony in your projects. 
            \item Color matching systems such as Web-safe colors and PMS make it easier to accurately reproduce colors no matter where they are viewed or where they are printed.
        \end{itemize} 

        \pagebreak \bigbreak \noindent
        \begin{Large}
            \noindent \textbf{Understanding RGB, CMYK,  and HSB:}
        \end{Large}
        \bigbreak \noindent
        One of the first things to decide when working with color in a digital media project is what color model to use. A color model is a group of colors identified in a way that computers can understand. So, instead of names like “red” or “blue,” a color model identifies colors with a combi-nation of numbers, kind of like a code. There are several different color models to choose from, but the three most common color models in digital media are RGB, CMYK, and HSB
        
        \bigbreak \noindent \bigbreak \noindent 
        \nt{If you are making something that will be viewed on a screen, such as a website, you will use RGB. If you are creating something that will be sent to a printer, like a book, you should choose CMYK}

        \bigbreak \noindent \bigbreak \bigbreak \noindent 
        \begin{large}
            \noindent \textbf{The RGB Color Model:}
        \end{large}
        \bigbreak \noindent
         To understand how the color models work, you’ll need to recall some concepts from art and science classes. At some point, you likely learned that primary colors are basic colors that cannot be created by mixing other colors. You also learned that mixing two primary colors creates a secondary color.

        \bigbreak \noindent
        Light is made up of different wavelengths, and, in what is called the visible spectrum, some of these wavelengths are associated with colors. All of these wavelengths combined create pure white light. But if certain wavelengths are blocked, humans perceive different colors. If all of these wavelengths are blocked, humans see black. This is called additive color mixing.

        \bigbreak \noindent
        In the RGB color model, three primary colors of light—red, green, and blue—are mixed at different intensities to create a range of other colors and display them on a computer screen. When the three colors of light mix in different combinations, they create different colors.

        \bigbreak \noindent
        In the RGB color model, each primary color is assigned a number between 0 and 255. This number tells the computer how much light should come through that primary color; 0 indicates no light at all and 255 indicates full light. Together, the numbers assigned to the three primary colors make up what is called an RGB triplet.

        \bigbreak \noindent \bigbreak
        \begin{large}
            \noindent \textbf{The CMYK Color Model:}
        \end{large}
        
        \bigbreak \noindent
         The CMYK color model produces colors by mixing color pigments, like ink or paint. When printed on material such as paper, ink actually absorbs light, blocking certain wavelengths from being reflected back to your eye. You see the reflected wavelengths as color. Mixing pigments together in dif-ferent amounts changes how much light is absorbed and creates a whole range of colors. This is called subtractive color mixing.

         \bigbreak \noindent
         In the CMYK color model, three primary pigment colors—cyan, magenta, and yellow—plus black mix in different combinations to create all of the CMYK colors. (The “K” in CMYK stands for “key”; the key color to which the other colors align during printing is black.) The scale for each pigment is 0 to 100 percent, with 0 percent being none of that color and 100 percent being the maximum amount. 

         \bigbreak \noindent \bigbreak
         \begin{large}
             \noindent \textbf{HSB, HSL, and HSV Color Models:}
         \end{large}

         \bigbreak \noindent
         software engineers developed a color model based on these qualities. In some software applications this is named the HSB color model for hue, saturation, and brightness. Some programs name it HSV for hue, saturation, and value. And others name it HSL for hue, satura-tion, and luminance. Unfortunately, the definitions for brightness, value, and luminance are not standardized across software and there can be some slight technical differences among the three

         \bigbreak \noindent
         In the HSB color model, hue refers to the general color of an object. Within the HSB model, hue is expressed as a degree between 0 and 360. Saturation refers to the intensity of a hue. Saturation varies on a scale from 0 to 100 percent, beginning with no color at all (gray) and ending with full color. Brightness refers to how light or dark a color is. Brightness varies on a scale from 0 to 100 percent, beginning with black and ending with white. The combination of these elements is what defines a color in the HSB color model.
         
         \bigbreak \noindent \bigbreak \noindent 
         \begin{large}
           \textbf{Selecting The Appropriate Color Model:}
         \end{large}
         \bigbreak \noindent
           Now that you’ve learned a little more about how RGB and CMYK colors are produced, it should be clear which color model matches which type of project. You use RGB for projects that will be viewed on an electronic screen such as a website, a slideshow presentation (Figure 5.5), or a video because RGB uses light to create color and an electronic display has a light source. You use CMYK for any project that is destined to be printed on a commercial press, such as a book, a magazine, a poster, or a bro-chure because CMYK uses pigments like ink or paint to apply color to a printed material like paper, canvas, plastic, etc. It’s important to make the right choice for a few reasons:
         \bigbreak \noindent 
         \begin{itemize}
            \item If you send a file that uses RGB colors to the printer, it won’t work. The colors in the file will have to be converted to CMYK to be printed. 
            \item The range of colors that a color model can produce is called a gamut. There are many more colors in the RGB gamut than in the CMYK gamut. This means it is possible to produce RGB colors that do not exist in CMYK. This is called out of gamut. If you choose RGB colors and later have to convert them to CMYK, you risk hav-ing colors that are out of gamut. In that case, you would have to choose new colors. 
            \item if you convert an RGB color to CMYK, the color may not be what you were expecting. You can avoid these problems by selecting and working with the proper color model in the first place.
            \item Choosing RGB color for files that will be viewed on a screen is important, too. RGB files are normally smaller than CMYK files and load onto screens faster.
         \end{itemize}
         \bigbreak \noindent 
          Once you determine which color model best suits your project, you indicate your choice by selecting the color mode in the digital media software you are using to create your piece. The color mode tells the computer which color model to use to represent colors. 

          \bigbreak \noindent \bigbreak \noindent 
          \begin{Large}
            \noindent \textbf{Recognizing Color Theory Terms and Concepts:}
          \end{Large}
          \bigbreak \noindent 
           Color theory is a set of guidelines about how colors communicate feelings and how they can be combined to create the best look and feel for a project.
          \bigbreak \noindent \bigbreak \noindent 
          \begin{large}
            \textbf{The Color Wheel and Basic Color Terminology:}
          \end{large}
          \bigbreak \noindent 
            You can begin exploring color theory by thinking about the color wheel. A color wheel begins with three primary colors. As mentioned earlier in the chapter, when two primary colors are mixed, the result is a secondary color, which is placed between the primary colors on a color wheel. When a primary color and secondary color mix, the result is a tertiary color
          \bigbreak \noindent 
           In color theory, colors on a color wheel are referred to as hues. Hue is basically just another word for color and you can use the terms interchangeably
          \bigbreak \noindent \bigbreak \noindent  
          \nt{it’s important to note that black, white, and gray are not hues in color theory; they are neutral colors. Neutrals are mixed with hues to change the nature of a color.}
          \bigbreak \noindent 
           A hue mixed with black creates a shade of that color. Navy blue is a shade of blue. A hue mixed with white produces a tint of that color. Pink is a tint of red.
          \bigbreak \noindent 
           When gray is mixed with a hue, the result is a tone of that color. A range of grays, as in a black and white photo for example, is called grayscale.
          \bigbreak \noindent 
           On the \textbf{\textit{color wheel}}, Hues opposite each other on the wheel are called complementary colors, <BS>" Hues next to each other on a color wheel are called analogous colors. Hues next to each other on a color wheel are called \textbf{\textit{analogous colors}}

          \bigbreak \noindent \bigbreak \noindent 
          \begin{Large}
            \textbf{Color and It's Meaning:}
          \end{Large}
          \bigbreak \noindent 
           Colors can produce different impressions on the people who view them. Just what impression a color makes on a person can vary based on age, gender, culture, and personal experience. However, color theory indicates certain colors and color combinations seem to cause similar reactions in most people
          \bigbreak \noindent 
           Certain hues are also associated with certain emotions or states of being.

          \bigbreak \noindent \bigbreak \noindent 
          \begin{Large}
            \textbf{Color Themes:}
          \end{Large}
          \bigbreak \noindent 
           Normally when you design something using color, you will use a color theme. Very simply put, a color theme is a combination of colors. The goal of a color theme is to create color harmony.
          \bigbreak \noindent 
           Color harmony occurs when a group of colors create a cohesive and pleasing combination. But randomly grouping colors doesn’t always create color harmony.
          \bigbreak \noindent 
          
          \bigbreak \noindent \bigbreak \noindent 
          \begin{large}
            \textbf{Monochromatic Color Theme:} 
          \end{large}
          \bigbreak \noindent 
           A monochromatic color theme includes a single color combined with shades and tints of that color. It is the most subdued color theme and doesn’t offer a lot of contrast in a design.
          \bigbreak \noindent 
           A mon-ochromatic color theme is easy to create and can make designs look elegant and calm.

          \bigbreak \noindent \bigbreak \noindent 
          \begin{large}
            \textbf{Complementary Color Theme}
          \end{large}
          \bigbreak \noindent 
           At its simplest, a complementary color theme includes two colors that sit directly across from one another on a color wheel.
          \bigbreak \noindent 
           To make a design more interesting, you can expand a complementary color theme by using different shades, tints, and tones of the two complementary colors in the theme.

          \bigbreak \noindent \bigbreak \noindent 
          \begin{large}
            \textbf{Analogous Color Themes:}
          \end{large}
          \bigbreak \noindent 
           The analogous color theme includes colors that are next to one another on a color wheel. For example, blue, blue-green, and green from the traditional color wheel in Figure 5.7 are an analogous color theme.
          \bigbreak \noindent 
          analogous color themes create a sense of calm and unity. However, they don’t have as much contrast as complementary color themes. To get the best effect, you can choose a main color and use the others as accents and/or vary the shades and tints of the hues in the theme to create contrast and energy in your designs.

          \bigbreak \noindent \bigbreak \noindent 
          \begin{large}
            \textbf{Triadic Color Themes:}
          \end{large}
          \bigbreak \noindent 
          The triadic color theme is a little trickier to create, but also a little more unexpected than the other traditional color themes. The triadic color theme includes three evenly separated hues from a color wheel. For example, the primary colors red, yellow, and blue are a triadic color theme because they are each separated by three other colors on the traditional color wheel.

          \bigbreak \noindent \bigbreak \noindent 
          \begin{large}
            \textbf{Color Theme Tools:}
          \end{large}
          
          \bigbreak \noindent 
          While you can develop your own themes manually in many digi-tal media software programs, you can also automatically generate color themes using online tools. For example, Adobe Color is a free, Web-based color tool you can use to automatically generate a color theme by selecting a base color and then selecting a color rule such as complementary or triadic.
          \bigbreak \noindent 
          Alternatively, you can browse through color themes created by other Color users.

          \bigbreak \noindent \bigbreak \noindent 
          \begin{Large}
            \textbf{Using Color Matching Systems:}
          \end{Large}
          \bigbreak \noindent 
          As you select color for your projects, keep in mind that a carefully selected color or set of colors may not appear the same in different viewing situations
          \bigbreak \noindent 
          There are some very sophisticated tools in digital media design software to help make sure that the colors you select are the colors your ultimate viewers will see in a project. These color systems help you predict how the colors you choose will look in the finished product.

          \bigbreak \noindent \bigbreak \noindent 
          \begin{large}
            \textbf{Web-Safe Colors:}
          \end{large}
          \bigbreak \noindent 
          At one point, Web designers were very concerned about color match-ing because not as many computer displays were able to show the wide range of colors that most are able to display today. This meant that colors chosen on a more powerful computer system would look much different when viewed on other less advanced systems. To address this concern, designers and developers experimented with RGB colors and came up with the Web-safe color set.
          \bigbreak \noindent 
          \nt{As computer video cards became more powerful and as displays using true color (32 bit) or high color (24 bit) became more common, the use of “safe” colors became less of a concern. Still, some digital media design programs have a feature that automatically limits the color choices to Web-safe colors.}

          \bigbreak \noindent \bigbreak \noindent 
          \begin{large}
            \textbf{Pantone Matching System:}
          \end{large}
          \bigbreak \noindent 
          While concern about color accuracy is subsiding somewhat in the Web design community, it’s still an issue for print media designers. Colors on a computer monitor (which are represented by RGB colors) may not match what is produced on a professional printing press using CMYK inks
          \bigbreak \noindent 
          To address concerns like this, standardized color sys-tems and associated ink formulations have been developed to identify and label specific colors in the design and printing industry. The colors in these systems are called spot colors
          \bigbreak \noindent 
          The most widely known color matching system is the Pantone Matching System, more commonly referred to as PMS color. There are more than 1,000 PMS colors. When you designate a spot color based on a Pantone color, a printer uses a single ink that matches that color rather than creating the color using the CMYK process.
          \bigbreak \noindent 
          \textbf{ Here are a couple of things to keep in mind when choosing to work with spot colors:}
          \begin{itemize}
            \item Use a printed PMS swatch book to choose your colors, You can’t rely on the monitor to show PMS colors accurately
            \item Because a printer has to add a specific plate to print spot color, it can be expensive.
          \end{itemize}


     
\end{document}
