\documentclass{report}

\input{~/dev/latex/template/preamble.tex}
\input{~/dev/latex/template/macros.tex}

\title{\Huge{Chapter 4 Notes: Digital Photography}}
\author{\huge{Nathan Warner}}
\date{\huge{Jan 29, 2023}}

\begin{document}
    \maketitle
    \begin{Large}
        \noindent \textbf{Learning Outcomes:}
    \end{Large}

    \bigbreak \noindent 
    \begin{itemize}
        \item Demonstrate knowledge of camera basics
        \item Transfer images from a digital camera to a computer 
        \item Use good composition skills to take photographs
        \item Photograph images with several different techniques
        \item Use camera settings to modify an image
        \item Adjust camera’s white balance to improve an image 
    \end{itemize}

    \bigbreak \noindent \bigbreak \noindent 
    \begin{Large}
        \textbf{Key Terms:}
    \end{Large}

    \bigbreak \noindent 
    \begin{itemize}
        \item \textbf{aperture:} Indicates the size of a camera lens’s opening. 
        \item \textbf{back lighting:} When light shines from behind the subject toward the cam-era, often casting all of the front details of a subject in shadow or silhouett.
        \item \textbf{candid photograph:} Non-posed, usually informal pictures
        \item \textbf{color temperature:} Measures the type of light shining on an image.
        \item \textbf{cropping:} In photo composition, including all wanted elements in a photo and excluding all unwanted element.
        \item \textbf{depth of field:} Indicates how much of the image is in focus.
        \item \textbf{digital zoom:} The digital enlargement of an image on an image sensor through interpolation of pixels
        \item \textbf{DSLR:} A camera that uses a mirror system to capture an image. DSLR cameras have interchangeable lenses.
        \item \textbf{focal point:} The element within an image on which the viewer’s eye focuse.
        \item \textbf{framing:} The use of elements within a scene to visually surround the subject and draw attention to it.
        \item \textbf{front lighting:} When light shines from behind the cam-era and illuminates the front of the subject, producing few or no shadow.
        \item \textbf{illusion of depth:} In photography, the effect of visual clues that make a viewer perceive an image as three-dimensional in a two-dimensional imag.
        \item \textbf{ISO:} A standardized mea-surement of the speed with which a camera stores an imag.
        \item \textbf{jog dial:} A type of wheel or dial on  a camera that makes it possible to scroll through setting options by rotating. There is an indicator for the current selection
        \item \textbf{Kelvin:} The measurement, in degrees, of the color tem-perature of light.
        \item \textbf{leading lines:} Actual or sug-gested lines in an image that draw a viewer’s eye through an image in a specific direc-tion, usually to the focal poin.
        \item \textbf{macro:} A setting or lens that allows close-up.
        \item \textbf{megapixels:} A unit of mea-sure equal to one million pixels. On a digital camera, the megapixels on an image sensor react to and record light to produce an image. The number of megapixels indicates the maximum image resolution of a camer
        \item \textbf{metadata:} Information about a photograph stored within the image file.
        \item \textbf{mode dial:} A type of wheel or dial on  a camera that makes it possible to scroll through setting options by rotating. There is an indicator for the current selection
        \item \textbf{optical zoom:} The actual magnification of an image through the movement of a camera lens
        \item \textbf{perspective:} In photog-raphy, what makes items look larger and closer or smaller and farther away; can be used to create depth and express a story about a subject.
        \item \textbf{photographic composition:} The selection and arrange-ment of design elements within a photograp.
        \item \textbf{point and shoot:} A camera designed to be easy to use with preset functions
        \item \textbf{rule of thirds:} Principle of imposing an imaginary grid of nine equal spaces (like a tic-tac-toe grid) over a scene to be photographed, then positioning the most impor-tant elements of the image along the gridlines, most preferably at or near the intersection of two imagi-nary gridline.
        \item \textbf{shutter speed:} Measures the rate at which a camera lens opens and closes.
        \item \textbf{side lighting:} Light that shines from the right or left of the camera and illumi-nates the subject from the side, creating more defined highlights and shadow.
        \item \textbf{white balance:} Adjusts an image based upon the color temperature present when an image is photographed.
    \end{itemize}

    \bigbreak \noindent \bigbreak \noindent 
    \begin{Large}
        \textbf{Key Concepts:}
    \end{Large}

    \bigbreak \noindent 
    \begin{itemize}
         \item Digital cameras fall into two categories: point and shoot and DSLR. 
         \item Some people choose DSLR cameras for high image quality, more adapt-ability, faster performance, and more manual control as compared to point-and-shoot cameras. 
         \item Point-and-shoot cameras are generally less expensive, less bulky, and easier to use than DSLR cameras. 
         \item wide range of shooting modes on a point-and-shoot camera can help you take better pictures in a bigger variety of situations. 
         \item Cameras measure the possible image size by using megapixels. You should consider how you will use images when determining how many megapixels you need in a camera. 
         \item In optical zoom, the camera lens actually moves to magnify a subject. It has no impact on the image resolution. In digital zoom, the image pro-cessor crops the area around the subject and uses interpolation to make it appear bigger. This can affect resolution in the final product.
         \item Most digital cameras save images as JPGs, which are relatively small and easy to use in a variety of settings, but use lossy compression. Some cameras offer a raw data file option, which is stable and offers more control over set-tings, but must be converted before images are editable and usable.
         \item Digital cameras must be treated carefully to prevent damage and maintain safety. 
         \item Photographic composition requires the photographer to consider the focal point, the rule of thirds, framing, leading lines, and depth.
         \item Shutter speed, aperture, and ISO determine the way a photograph is  captured. 
    \end{itemize}

    \pagebreak
    \begin{Large}
        \noindent \textbf{Reviewing Cameras:}
    \end{Large}

    \bigbreak \noindent 
    In an earlier time, almost all cameras used film to capture images. To produce a photograph, the film had to be developed using special chemicals and then printed on photographic paper. Even in the early days of digital media, to convert images into a digital format, you had to scan hardcopy photographs using an optical scanner connected to a computer. But advances in technology and the introduction of  digital cameras dramatically changed the process. In digital cameras, electronic image sensors capture images and built-in computer  processors digitize image information, saving it to a memory device such as a flash memory card

    \bigbreak \noindent 
    Digital cameras offer many benefits over conventional film cameras. Digital cameras allow you to review your images immediately on a preview screen instead of waiting for film to be developed and printed. Also, a typical memory card can store many more images than a roll of film, can be erased and reused many times, and eliminates process-ing costs to produce photos. Therefore, taking a lot of pictures with a digital camera is far less expensive than with a film camera

    \bigbreak \noindent 
    \nt{Some professional photographers, especially fine art photographers, still use film cam-eras, arguing that film offers image quality that even the most advanced digital cameras can’t provide}

    \bigbreak \noindent \bigbreak \noindent 
    \begin{Large}
        \textbf{DSLR Versus Point and Shoot:}
    \end{Large}

    \bigbreak \noindent 
    Digital cameras fall into two basic categories: DSLR (digital single lens reflex) and point and shoot. Generally, DSLR cameras are designed for the professional or serious amateur photographer who is most inter-ested in image quality. Point-and-shoot cameras are designed for the more casual photographer who is looking for ease of use and convenience

    \bigbreak \noindent 
    Even simpler to use than point-and-shoot cameras are the cameras built into the smartphones and tablets that have become ubiquitous in the past few years. These cameras do not allow for the adjustment of many settings, but they can record a scene quickly and conveniently, and the resulting image can be easily transmitted using the phone’s message or email feature.

    \bigbreak \noindent 
    People who choose DSLR cameras over point-and-shoot cameras commonly note the following advantages:

    \bigbreak \noindent 
    \begin{itemize}
        \item Better image quality 
        \item More adaptability
        \item Faster performance
        \item More manual contral
        \item What you see if what you get
        \item Less expensive
        \item More portable
        \item Easier to use
    \end{itemize}

    \bigbreak \noindent 
    \begin{Large}
        \textbf{Camera Features:}
    \end{Large}
    
    \bigbreak \noindent 
    Each brand of digital camera has its own features. The best way to learn about your camera’s particular features is to read the owner’s manual or search the Web for additional information.

    \bigbreak \noindent \bigbreak \noindent 
    \begin{Large}
        \textbf{Shooting Modes:}
    \end{Large}

    \bigbreak \noindent 
    Most point-and-shoot cameras offer several shooting mode options beyond auto mode. Depending on the shooting mode you choose, the camera automatically adjusts the settings to capture the best possible image.

    \bigbreak \noindent 
    The different shooting modes are often represented by icons on a small dial called a jog dial or mode dial.

    \pagebreak
    \begin{Large}
        \noindent \textbf{Megapixels:}
    \end{Large}

    \bigbreak \noindent 
    The image sensor in a digital camera is made up of millions of pixels that react to and record light to capture a digital image. In fact, there are so many pixels on an image sensor, they are referenced in terms of megapixels (1 megapixel = 1 million pixels). The number of megapix-els on the image sensor directly impacts the potential resolution of the photos a camera captures. In general, the greater the number of pixels, the higher the level of detail and the better the quality of the photograph. The first popular digital cameras had 2 or 3 megapixels. These cameras were fine for viewing images on a screen or even for printing up to 4- × 6-inch prints. However, if you printed anything larger than that, the lack of resolution showed. In the early days, the number of megapixels in a camera was an important consideration when decid-ing what model to buy.As higher megapixel cameras became available, consumers had to weigh their need for higher resolution photos against higher price tags. Fortunately, technology advanced rapidly, megapixels went up, and camera prices came down. Now, a 3-megapixel (or lower) camera is rare, except perhaps in a cell phone.

    \bigbreak \noindent 
    The question these days may be how many megapixels is too much. More megapixels mean bigger file sizes. If you don’t intend to make large prints of your photos, you probably don’t need a high megapixel camera

    \bigbreak \noindent \bigbreak \noindent 
    \begin{Large}
        \textbf{Optical Versus Digital Zoom:}
    \end{Large}

    \bigbreak \noindent 
    With optical zoom, the camera lens physically moves to magnify your subject and make it appear closer, similar to the function of lenses in binoculars. Optical zoom has no impact on the resolution of the image—the pixels are unchanged by zooming. In digital zoom, the digital image sensor crops the area around your subject and digitally enlarges the subject to fill the frame. To make the subject bigger, the camera inter-polates (or makes up) pixels based on the information in existing pix-els and adds them to the image. The result can be fuzzy or distorted, especially if you eventually enlarge and print the image.

    \bigbreak \noindent 
    Newer cameras offer what is sometimes referred to as intelligent or smart digital zoom. With intelligent zoom, instead of interpolating pixels and enlarging the image to fill the entire available image space, the camera simply crops the information around the subject and does not enlarge it. This means that you can digitally zoom in on a subject without losing the photo quality, but your finished image will be smaller than the normal picture size.

    \bigbreak \noindent 
    \nt{Many photographers simply turn off the digital  zoom, smart or otherwise, and rely solely on optical zoom for  their photographs.}

    \bigbreak \noindent \bigbreak \noindent 
    \begin{Large}
        \textbf{File Types:}
    \end{Large}

    \bigbreak \noindent 
    The default file format for most digital cameras is JPEG (.jpg). JPEG files are often the best file type for digital photographs because they are relatively small and widely accepted by photo editing and printing software and hardware. In addition, many digital cameras allow you to set the quality of the JPEG as high, medium, or low. Higher quality JPEGs take up more memory but give you better results

    \bigbreak \noindent 
    \nt{One drawback of JPEG files is that they use lossy compression, meaning they lose a little bit of digital information each time they are manipulated and saved.}

    \bigbreak \noindent 
    some professional or serious amateur photogra-phers prefer cameras that can save an image in a raw file format. Raw image files include all of the image data without any compression or processing. Raw image files must be processed by the photographer using conversion software on a computer. However, for most digital media projects, small, easy-to-use JPEG files are a good choice.

    \pagebreak
    \begin{Large}
        \noindent \textbf{Camera Care and Safety:}
    \end{Large}

    \bigbreak \noindent 
    Follow these rules to keep your camera in good working order:

    \bigbreak \noindent 
    \begin{itemize}
        \item Keep the lens area clean by using a soft cloth designed especially for camera lenses.
        \item Keep the lens cap on or the shutter closed (usually by turning off the camera) when not using the camera.
        \item Beware of dropping the camera even from a short distance such as onto a counter.
        \item Never force a memory card in or pull it out if there is resistance.
        \item Never force an uncooperative switch; instead find out why the latch or door is unable to function.
        \item Never throw rechargeable camera batteries in the trash: They can leak toxins into the environment. Do a simple search online to find out where you can recycle rechargeable batteries.
        \item Don’t shoot directly into the sun, to avoid damaging your eyes through the viewfinder.
        \item Don’t walk while looking through the viewfinder, to avoid tripping over or running into obstacles.
        \item Be careful with a camera’s batteries; don’t store them where they can come in contact with each other, and don’t handle batteries that appear to be leaking.
        \item Don’t try to fix a digital camera yourself by opening the camera; some components in digital cameras are capable of delivering an electrical shock.
    \end{itemize}

    \bigbreak \noindent \bigbreak \noindent 
    \begin{Large}
        \textbf{Taking Photographs:}
    \end{Large}

    \bigbreak \noindent 
    Candid photographs are unplanned photographs taken with little or no preparation or posing. While these are fun to take and post on a social networking site, tak-ing good photographs requires some thought and preparation. With a little bit of knowledge, even your candid shots can improve significantly.

    \bigbreak \noindent 
    Taking great photographs requires an understanding of the elements of design and the principles of photographic composition

    \bigbreak \noindent 
    Elements of design include:

    \bigbreak \noindent 
    \begin{itemize}
        \item line
        \item shape
        \item color
        \item lighting
    \end{itemize}

    \bigbreak \noindent 
    Principles of composition include:

    \bigbreak \noindent 
    \begin{itemize}
        \item unity
        \item balance
        \item perspective
        \item emphasis
    \end{itemize}

    \bigbreak \noindent 
    \nt{You can think of elements as the build-ing blocks and composition as the arrangement of the blocks to create a final product.}

    \bigbreak \noindent \bigbreak \noindent 
    \begin{Large}
        \textbf{Focal Point:}
    \end{Large}

    \bigbreak \noindent 
    A focal point is the most important element on which you want a viewer to focus. Once you choose a focal point, you can use some simple principles of composition to organize the other elements in the image to direct your viewer’s attention to the focal point of the image.

    \bigbreak \noindent \bigbreak \noindent 
    \begin{Large}
        \textbf{Rule of Thirds:}
    \end{Large}
    
    \bigbreak \noindent 
    You may think that placing the focal point in the center of a photo-graph is the best way to draw attention to it. Surprisingly, this is not necessarily true. On a rectangular surface, the human eye naturally focuses on an off-center spot.

    \bigbreak \noindent 
    The rule of thirds states that an image should be divided into an imaginary grid of nine equal parts and that the focal point should be placed at or close to the point where the lines of the grid intersect

    \bigbreak \noindent 
    To apply the rule of thirds, imagine a tic-tac-toe grid placed over a scene, dividing it into three columns and three rows. Place the focal point along the imaginary lines, preferably at or near the spot where a horizontal and a vertical grid line meet—that is, off center. Your viewer’s eye naturally moves to an area slightly off center and rests there. Because it feels natural, the viewer finds the image pleasing

    \bigbreak \noindent \bigbreak \noindent
    \begin{Large}
        \textbf{Framing:}
    \end{Large}

    \bigbreak \noindent 
    Another composition technique you can use to draw attention to your focal point is framing. Framing is using elements in a scene to visually surround your subject and make it stand out. So, for example, a wedding photographer might photograph the bride and groom in the doorway of a church. The outline of the doorway forms a visual frame around the happy couple and makes them stand out.

    \bigbreak \noindent 
    \nt{One key to framing is to make sure that the frame does not draw too much attention to itself since the objective is to focus on your subject.}

    \bigbreak \noindent \bigbreak \noindent 
    \begin{Large}
        \textbf{Leading Lines:}
    \end{Large}

    \bigbreak \noindent 
    As the name suggests, leading lines are visual elements in a photograph that draw a viewer’s eye through an image in a specific direction. Because the human eye naturally follows a line, whether literal or suggested, you can use lines in your photo compo-sition to draw a viewer’s eye through the photograph and to the focal point.

    \bigbreak \noindent \bigbreak \noindent 
    \begin{Large}
        \textbf{Cropping:}
    \end{Large}

    \bigbreak \noindent 
    Cropping in photo composition means including all of the elements you want in the picture and excluding anything you don’t want. When you take a picture, you may be so focused on the main subject that you do not notice other distracting elements in the viewfinder.
    
    \bigbreak \noindent 
    Photographers sometimes talk about “filling the frame,” meaning that the subject—and only the subject—occupies the limited space of the photo. Cropping means you are filling the frame, eliminating unwanted elements, and allowing your subject to occupy more space. This can make a photo more interesting and dramatic and leaves no doubt about what you were trying to capture when you took the picture.

    \pagebreak
    \begin{Large}
        \noindent \textbf{Directional Lighting:}
    \end{Large}

    \bigbreak \noindent 
    Lighting is perhaps one of the most important elements of a photograph and also one of the most complicated to explain. The color, direction, amount, and quality of light all have an impact on photographs

    \bigbreak \noindent 
    a good place to start learning about light in photographs is to consider how the direction of light falling on your subject affects Digital Photographyphoto composition by contrasting light and shadow. This contrast helps a viewer perceive a subject as three-dimensional and interesting.

    \bigbreak \noindent 
    There are three basic types of directional lighting:

    \bigbreak \noindent 
    \begin{itemize}
        \item Front
        \item Side
        \item Back
    \end{itemize}

    \bigbreak \noindent 
    Front lighting comes from behind the camera and illuminates the front of the subject. It can come from many sources, including the sun, indoor lights, and the flash on your camera. Photos shot with front lighting are generally bright and well lit, but this type of lighting produces the fewest shadows and can make your subject look flat in photographs.

    \bigbreak \noindent 
    In side lighting, the light is directed at a subject from either the left or the right side, creating lots of shadows and highlights. Side lighting emphasizes texture and gives depth to your subject, so you may find it the most useful for delivering high-impact, interesting photos

    \bigbreak \noindent 
    In back lighting, the light shines from behind the subject, toward the camera. It emphasizes the shape and outline of a subject, but other details are mostly lost in shadow. At its most extreme, back lighting produces a silhouette of your subject.

    \bigbreak \noindent \bigbreak \noindent 
    \begin{Large}
        \textbf{Depth:}
    \end{Large}

    \bigbreak \noindent 
    In photography, the camera takes three-dimensional scenes and records them on the flat, two-dimensional surface of a photograph. Even though there is no third dimension of actual depth in a photograph, the brain uses the same visual clues in a photograph to perceive the illusion of depth. The more a viewer perceives depth in a photo, the more lifelike and interesting it is.

    \bigbreak \noindent \bigbreak \noindent 
    \begin{Large}
        \textbf{Perspective:}
    \end{Large}

    \bigbreak \noindent 
    Perspective in photography is what makes items look bigger, smaller, closer, and farther. You can manage perspective in your photo composition to give an image depth. Perspective in a photo can be affected by special lenses if you have a DSLR camera. Even some point-and-shoot camera settings can mimic different lenses to change perspective in photos.

    \bigbreak \noindent 
    Moving closer, farther away, or even to the side of your subject changes its perceived relationship to the other items in the scene and the perspective of the image. Note that this is not the same as zoom-ing in on a subject. To change the perspective, you have to change the physical position of the camera in relation to the subject.

    \bigbreak \noindent \bigbreak \noindent 
    \begin{Large}
        \textbf{Angles:}
    \end{Large}

    \bigbreak \noindent 
    Just as you can change the perspective of an image by moving closer, farther, or to the side of your subject, you can impact perspective and depth by changing the angle of the shot. Beginner or casual photogra-phers often shoot their subjects from a standing, eye-level position. This position is considered the normal perspective, but it often lacks depth. However, simply changing the angle from which you shoot a subject can completely alter the perspective and the impression your photo makes on a viewer.

    \pagebreak
    \begin{Large}
        \noindent \textbf{Foreground, Middle Ground, Background}
    \end{Large}

    \bigbreak \noindent 
    Most snapshots have just two dimensions: the subject and whatever is behind it. It’s easy to add depth to these shots simply by adding another element in front of your main subject. Doing so creates the illusion of three dimensions: the elements behind your sub-ject are in the background, the subject is in the middle ground, and the elements in front of your subject are in the foreground.

    \bigbreak \noindent \bigbreak \noindent 
    \begin{Large}
        \textbf{Adjusting Settings:}
    \end{Large}

    \bigbreak \noindent 
    A camera uses three components to capture an image. They consist of the shutter speed (how fast the lens opens and closes), aperture settings (how wide the lens opens to let in light), and the  ISO (speed at which the “film” captures an image).All three of these functions work together in combination to produce an image. Changing any of these three settings will change the way your photograph appears.

    \bigbreak \noindent 
    \nt{Many cameras provide pre-set special image types such as a macro(close-up) setting, a distance setting, or a sports setting. These special settings use different aperture, shutter speed, and ISO to capture the moment}

    \bigbreak \noindent \bigbreak \noindent 
    \begin{Large}
        \textbf{Shutter Speed:}
    \end{Large}

    \bigbreak \noindent 
    You can adjust the shutter speed to achieve specific and desired effects. For example, if your shutter speed is set to open and close slowly while you are taking a picture of a runner crossing the finish line, there is enough time to capture a series of movements, resulting in the runner’s blur. If you have the shutter speed set to open and close briefly, then you may catch the specific instant when the runner’s foot is coming down onto the finish line

    \bigbreak \noindent \bigbreak \noindent 
    \begin{Large}
        \textbf{Aperture:}
    \end{Large}

    \bigbreak \noindent 
    Changing the aperture settings determines the depth of field. Aperture settings are meas-ured in f-numbers (often referred to as f-stop or f-ratio). A larger f-stop value allows less light in. A smaller f-stop value lets in more light

    \bigbreak \noindent \bigbreak \noindent 
    \begin{Large}
        \textbf{ISO:}
    \end{Large}

    \bigbreak \noindent 
    Even though digital cameras have no film, ISO is still used to indicate how the image is captured to the storage media. Low ISO settings capture an image sharply with few extra pixels or graininess. Low ISOs are used when there is plenty of light. High ISO settings work hard to capture an image in low light but the result is an image that has artifacts.

    \bigbreak \noindent 
    \nt{Artifacts are irregularities in digital images, such as jagged edges or pixelation}

    \bigbreak \noindent \bigbreak \noindent 
    \begin{Large}
        \textbf{White Balance:}
    \end{Large}

    \bigbreak \noindent 
    Cameras attempt to establish settings based upon the temperature of the light coming in, but it is not always accurate. As a result, one of the functions cameras provide is the option to set the white balance. White balance means that the camera attempts to make white look truly white without the yellowness of a candle or the blue of an overcast day. White balance settings ask you what type of light you are photographing in so that the camera does not have to guess

    \bigbreak \noindent \bigbreak \noindent 
    \begin{Large}
        \textbf{Metadata:}
    \end{Large}
    
    \bigbreak \noindent 
    Digital photographs have information “hidden” in their file struc-ture that can be viewed in some editing software. This information, called metadata, records a number of valuable details about the photograph, including the date and time when the image was taken, the type of camera used, and the various set-tings such as aperture and shutter speed.
\end{document}



