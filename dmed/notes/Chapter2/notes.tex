\documentclass{report}

\input{~/dev/latex/template/preamble.tex}
\input{~/dev/latex/template/macros.tex}

\title{\Huge{Chapter 2 Notes: Ethical and Legal Issues}}
\author{\huge{Nathan Warner}}
\date{\huge{Jan 29, 2023}}

\begin{document}
    \maketitle

    \begin{Large}
        \noindent \textbf{Learning Outcomes:}
    \end{Large}

    \bigbreak \noindent 
    \begin{itemize}
        \item Explain the concept of intellectual property, including copyright and trademarks
        \item Identify the difference between copyright violations and plagiarism
        \item Demonstrate proper use of citations and fair use
        \item discuss the ethical challenges facing digital media, including piracy and file sharing
        \item Explain how licensing applies to software
    \end{itemize}

    \bigbreak \noindent \bigbreak \noindent 
    \begin{Large}
        \textbf{Key Terms:}
    \end{Large}

    \bigbreak \noindent 
    \begin{itemize}
        \item \textbf{Copyleft:} A licensing pro-tection used by those who create open source software
        \item \textbf{Copyright:} The term literally means restricting the right of others to copy
        \item \textbf{Deep Linking:} Citing a Web address that goes beyond the home or entry page
        \item \textbf{Digital Rights Management (DRM):} Technology that prevents unauthorized  copying of a digital work
        \item \textbf{End-user License Agreement (Eula):} A contract soft-ware purchasers must agree to before using software
        \item \textbf{Ethics:} Moral choices between right and wrong actions
        \item \textbf{Fair Use:} The right to reproduce a small part of a copyrighted work for educa-tional or other not-for-profit purposes without having to obtain permission or pay a royalty fee
        \item \textbf{File Sharing:} Use of a  network to move files between computers,  often for illegal purposes.
        \item \textbf{GNU General Public License:} The standard open source contract or license
        \item \textbf{Intellectual Property:} A legal concept that protects a creative work just as if it were physical property
        \item \textbf{Open Source:} Software that allows others to use its code without cost.
        \item \textbf{Patent:} A property right for an invention that lasts a  specified period of time
        \item \textbf{Piracy:} Copying a product  (often digital) for profit without authorization from the owner. Music and video products as well as software are frequently subjects of pirating
        \item \textbf{Plagiarism:} Copying or otherwise using someone else’s creative work and claiming it as your own,  usually in an academic or journalistic work, but also more recently in social media
        \item \textbf{Propietary:} term used for software code that has restricted rights of use
        \item \textbf{Public Domain:} Creative works whose copyright restrictions have expired. The term may also be used for open source software
        \item \textbf{Royalty:} A fee paid to the person who owns the  copyright on a creative  work when it is used by someone else.
        \item \textbf{Royalty Free:} A type of licensing agreement that gives the buyer almost unlimited permission to use a copyrighted image for a one-time fee.
        \item \textbf{Social Media:} Websites that allow users to create and exchange information
        \item \textbf{Trademark:} A word, phrase, or image used to identify something as a product of a particular business
    \end{itemize}

    \pagebreak
    \begin{Large}
        \noindent \textbf{Key Concepts:}
    \end{Large}

    \bigbreak \noindent 
    \begin{itemize}
        \item Intellectual property is a creation of the mind that is treated as a tangible property. 
        \item Copyright laws give the owner of a creative work the legal right to restrict who may copy the work.
        \item trademark is a distinctive word, phrase, or image that visually identifies something as a product of a particular business or organization. A patent is a property right for an invention that lasts a specified period of time.
        \item Fair use allows students and some professionals to reproduce a small part of another person’s work, though proper credit should be given to the creator. 
        \item Plagiarism involves copying another person’s creative work and claiming it as your own. 
        \item Piracy involves stealing another person’s creative work, usually for profit 
        \item Ethical decisions require you to make choices about what is right.
    \end{itemize}

    \bigbreak \noindent \bigbreak \noindent \bigbreak \noindent 
    \begin{Large}
        \textbf{Safeguarding Intellectual Property:}
    \end{Large}

    \bigbreak \noindent 
    One of the ways creative works are protected is through the concept of intellectual property. Laws treat the creative works of a mind as real property even if you cannot see their physical presence.

    \bigbreak \noindent 
    if someone creates something using their mind that is not likely to be created in the same way by another person, then that piece of creativity is an intellectual property. If you take a photo-graph, the image itself rather than the print is intellectual property. If you write a term paper, the words you use to complete the project are intellectual property as soon as they appear on the page. No special action must be taken to establish protection for those words. If you compose a song, the notes and the words as they are joined together are the property of the composer regardless of the form the song takes. The song is yours whether you perform it or merely write it down.

    \bigbreak \noindent \bigbreak \noindent \bigbreak \noindent 
    \begin{Large}
        \textbf{Copyright:}
    \end{Large}

    \bigbreak \noindent 
    Copyright is a legal means of establishing ownership of an intellectual property. The term copyright actually comes from the two words “copy” and “right.” It means that the owner of a creative work has the legal right to restrict who may copy the work. Copyright does not prevent others from using the work, but instead it requires permission from the copyright owner in order to be used

    \bigbreak \noindent 
    Basically, as soon as the creative idea is shared with the world in some physical way, it becomes eligible for copyright. That means a poem in your head that is never shared with anyone is not eligible for copyright, but if you scribble it on a napkin for someone else to read, it is

    \bigbreak \noindent 
    \nt{A Copyright holds the format, Copyright © [date] [Name of Copyright Holder].}

    \bigbreak \noindent 
    A Copyright serves sereval purposes:

    \bigbreak \noindent 
    \begin{itemize}
        \item It reminds the user that the work is protected
        \item It clearly establishes the date when protection begins
        \item It makes it easier for someone to seek permission to use the work because the 
            owner is named
    \end{itemize}
    \nt{Copyright lasts for the life of the author plus an additional 70 years. Registration of a copyright requires sub-mitting a form and paying a fee to the U.S. 
    Copyright Office}

    \bigbreak \noindent \bigbreak \noindent 
    \begin{Large}
        \textbf{Permission:}
    \end{Large}

    \bigbreak \noindent 
    an important element  of copyright is the owner’s right to grant permission for someone else to use his or her creative property. You might wonder how to get per-mission to use copyrighted material that you plan to include in a work of your own. The answer is simple: Ask. A phone call, a letter, an email, perhaps even a text message, can be used to seek permission to use someone else’s intellectual property.
    
    \bigbreak \noindent \bigbreak \noindent 
    \begin{Large}
        \textbf{Royalties}
    \end{Large}

    \bigbreak \noindent 
    Sometimes, copyright holders grant permission to use their material for free. Other times, they charge a fee. When you pay to use someone’s copyrighted material, you are not paying for the object itself but rather you are paying for the permission, or license, to use it. Some copyright holders charge a fee for each use of the material. This fee is referred to as a royalty

    \bigbreak \noindent 
    \nt{When you see \textbf{ \textit{Royalty Free}}, These are images and audio files that you may purchase and use more than once without having to pay a fee each time they are published or used. Note that they are not necessarily free, despite the name.}

    \bigbreak \noindent \bigbreak \noindent 
    \begin{Large}
        \textbf{Software Licenses:}
    \end{Large}

    \bigbreak \noindent 
    Software programs written by companies such as Microsoft are consid-ered proprietary. This means that the software is owned by the com-pany that created it. When you purchase proprietary software, you are licensed to use it. You do not own it. When you install software on a computer, you must accept licensing limitations before installation can proceed. This is called the end-user license agreement (EULA)

    \bigbreak \noindent 
    Some software is developed as open source. This means that the code used to create the soft-ware is open to anyone who wants to make changes to it. Open source software is free but not without copyright protections. In a play on the word copyright, the term copyleft has been used to describe the legal protections for open source. The open source license allows users to use, study, copy, share, and modify the software. The most commonly used open source license is the GNU General Public License

    \bigbreak \noindent 
    File formats can also be considered proprietary if they were created using licensed code. Other software file formats are not proprietary and are considered part of the public domain. For word processors, this would be a .txt

    \bigbreak \noindent \bigbreak \noindent 
    \begin{Large}
        \textbf{Trademarks and Patents:}
    \end{Large}

    \bigbreak \noindent 
    One way of establishing intellectual ownership is to create a trade-mark. A trademark is a distinctive word, phrase, or image used to iden-tify something as a product of a particular business or organization

    \bigbreak \noindent 
    Trademarks that are registered carry the ® symbol while unregistered ones use the ™ symbol. While copyrights are administered by the U.S. Copy-right Office, trademarks are handled by the U.S. Patent and Trademark Office because they represent a business function

    \bigbreak \noindent 
    to make the ownership of intellectual property even more bind-ing, a person can patent his or her invention. A patent gives the pat-ent holder a property right to an invention

    \pagebreak
    \begin{Large}
        \noindent \textbf{Illegal File Sharing:}
    \end{Large}

    \bigbreak \noindent 
    However, the term file sharing has come to mean the illegal transfer of copyrighted mate-rial between computers. This is a form of copyright violation. Websites and software have been developed to make it easy to share files, but that does not make it okay.

    \bigbreak \noindent 
    many people violate legal boundaries is when they “share” a digital product that they have purchased. They assume that since they have paid for the product, they can share it with any-one they choose just as they might share a candy bar with a group of friends. This is untrue

    \bigbreak \noindent
    \nt{charges against file-sharing networks and individuals who illegally share copyrighted material. Even first-time offenders can face hundreds of thousands of dollars in fines and possible imprisonment if they ille-gally share files. This is true even if they did not know they were acting illegally or if they did not profit from the file sharing. It is important, therefore, that you learn the rules about sharing files before you upload or download anything}

    \bigbreak \noindent \bigbreak \noindent 
    \begin{Large}
        \textbf{Piracy:}
    \end{Large}

    \bigbreak \noindent 
    Making a copy of a software package, a video, a music CD, or a digital book is sometimes referred to as piracy, especially when the illegally copied item is sold for profit

    \bigbreak \noindent \bigbreak \noindent 
    \begin{Large}
        \textbf{Digital Rights Management (DRM):}
    \end{Large}

    \bigbreak \noindent 
    One method that businesses have used to prevent unauthorized copying is to control how a digital work can be used. Digital rights management (DRM) is a form of technology that controls digital copying by inserting a software program into the CD (or other media) that restricts copying. It often prevents the purchaser of a CD or video from making copies of the audio or video work even as a backup.

    \bigbreak \noindent 
    \nt{DRM is a direct result of piracy.}

    \bigbreak \noindent \bigbreak \noindent 
    \begin{Large}
        \textbf{Fair use Policy;}
    \end{Large}
    
    \bigbreak \noindent 
    Students are taught in school that they can use excerpts from another person’s work provided they give proper credit or documentation. This use of copyrighted material is called fair use. Fair use is generally limited to educational copying and is very limited in the commercial or business world

    \bigbreak \noindent 
    Section 107 in the United States copyright law outlines the four factors to consider when deciding on fair use:

    \bigbreak \noindent 
    \begin{itemize}
        \item The purpose and character of the use, including whether such use is of commercial nature or is for non-profit educational purposes
        \item The amount and substantiality of the portion used in relation to the copyrighted work as a whole
        \item The effect of the use upon the potential market for, or value of, the copyrighted work
        \item The distinction between fair use and infringement may be unclear and not easily defined. There is no spe-cific number of words, lines, or notes that may safely be taken without permission.
    \end{itemize}

    \bigbreak \noindent 
    \nt{Acknowledging the source of the copyrighted mate-rial does not substitute for obtaining permission}

    \bigbreak \noindent \bigbreak \noindent 
    \begin{Large}
        \textbf{Plagiarism:}
    \end{Large}

    \bigbreak \noindent 
    Plagiarism means using someone else’s work as your own and not giv-ing the original author credit. Plagiarism in the digital world is an easy trap to fall into. Copying words, capturing images, and adding some-one’s YouTube clip to a website are all done so frequently that we often do not think about the plagiarism this may involve. Plagiarism is different from copyright infringement. Schools and the academic world punish plagiarism. Courts address copyright issues

    \bigbreak \noindent 
    \nt{The way to avoid plagiarism is to document the source of your material, it is not enough to cite the source of copyrighted material to avoid copyright infringement. You cannot use copyrighted material in your own work. You must seek permission first.}

    \bigbreak \noindent \bigbreak \noindent 
    \begin{Large}
        \textbf{Using Proper Website Citations:}
    \end{Large}

    \bigbreak \noindent 
    Several pieces of information are usually required for a proper  citation of material you get from a website:

    \bigbreak \noindent 
    \begin{itemize}
        \item Author
        \item Date
        \item Title of Article
        \item Access Date
        \item URL
    \end{itemize}

    \bigbreak \noindent \bigbreak \noindent 
    \begin{Large}
        \textbf{Online Bibliographies:}
    \end{Large}

    \bigbreak \noindent 
    There are a number of websites (such as EasyBib and BibMe) that help you create bibliography entries. On these sites, you enter the Web address or the name of the print document and fill in detail boxes. Once the information is complete, you can then copy and paste into your research document a pre-formatted bibliography entry.

\end{document}
