\documentclass{report}

\input{~/dev/latex/template/preamble.tex}
\input{~/dev/latex/template/macros.tex}

\title{\Huge{Chapter 1 Notes: Getting Started With Digital Media}}
\author{\huge{Nathan Warner}}
\date{\huge{Jan 27, 2023}}

\begin{document}
    \maketitle
    \begin{Large}
        \noindent \textbf{Learning Outcomes:}
    \end{Large}

    \bigbreak \noindent 
    \begin{enumerate}
        \item List the characteristics needed to become a skilled digital master.
        \item Identify how to name and save a file.
        \item Explain how to ensure digital security. 
        \item Practice the techniques for good keyboarding.
    \end{enumerate}

    \bigbreak \noindent \bigbreak \noindent \bigbreak \noindent 
    \begin{Large}
        \textbf{Key Terms:}
    \end{Large}

    \bigbreak \noindent 
    \begin{itemize}
        \item \textbf{Adware:} Software that displays unwanted advertisements 
        \item \textbf{Cyber Predator:} A person who uses the Internet to make contact with others (usually with children and teens) in order to harm them
        \item \textbf{Digital Media:} any com-bination of audio, video, images, and text used to convey a message through technology
        \item \textbf{Encryption:} Converting text into an unreadable series of numbers and let-ters to protect information. Digital encryption uses soft-ware that can scramble and unscramble the data
        \item \textbf{Ergonomics:} A science that studies the best way to design a workplace for maximum safety and productivity
        \item \textbf{Hacker:} A person who finds an electronic means of gain-ing unauthorized access to a computer.
        \item \textbf{Keylogger:} Software that tracks keyboard use and transmits it to be used for illegal purposes
        \item \textbf{Malware:} The abbrevia-tion for malicious software, designed to damage a com-puter or steal information.
        \item \textbf{Naming Convention:} A set of rules used in the naming of files and folders
        \item \textbf{Online Backup:} A means of backing up or storing data using the Internet.
        \item \textbf{Phising:} A social engi-neering activity where the perpetrator uses fake websites or emails to trick a user into providing personal information or passwords
        \item \textbf{Repetitive Stress Injury:} Muscle or joint injury that results from performing actions repeatedly.
        \item \textbf{Rootkit:} Type of malware that hides its presence on a computer
        \item \textbf{Server:} A computer designed to store files from multiple computers
        \item \textbf{Social Engineering:} Tricking users into providing information in the belief that a request is legitimate.
        \item \textbf{Spyware:} Software that gathers information about a user without their knowledge
        \item \textbf{Trojan:} Type of malware that disguises itself as legitimate software
        \item \textbf{Virus:} Type of malware that replicates itself and spreads to other computers
        \item \textbf{Worm:} Type of malware that also replicates itself but primarily spreads through networks
    \end{itemize}

    \pagebreak
    \begin{Large}
        \noindent \textbf{Key Concepts:}
    \end{Large}

    \bigbreak \noindent 
    \begin{itemize}
        \item The five commitments to learning include: be flexible, keep an open mind, use initiative, listen and read attentively, and seek to acquire new knowledge and skills. 
        \item The six behaviors that contribute to your ability to acquire a job and grow in the field of your choice include good attendance, promptness, proper attire, a clean and safe work environment, appropriate voice, and pride. 
        \item You can demonstrate your digital media skills by seeking certification from a secondary and/or post-secondary school or through a provider such as Adobe or Microsoft.
        \item Managing digital files is an essential part of creating a good work environment. 
        \item Strong passwords are those that meet a set of rules designed to make it difficult for others to figure out the word. 
        \item Repetitive stress injury (RSI) (including carpal tunnel syndrome) results from repeated movement of a particular part of the body.
    \end{itemize}

    \bigbreak \noindent \bigbreak \noindent \bigbreak \noindent  
    \begin{Large}
        \noindent \textbf{Commitment:}
    \end{Large}
    \bigbreak \noindent 
    In order to learn new software and computer skills, you must:

    \bigbreak \noindent 
    \begin{itemize}
        \item Be flexible
        \item Keep an open mind
        \item Use initiative
        \item Listen and read attentively
        \item Seek to acquire new knowledge and skills
    \end{itemize}

    \bigbreak \noindent \bigbreak \noindent \bigbreak \noindent 
    \begin{Large}
        \textbf{Work Skills For Multimedia Careers:}
    \end{Large}

    \bigbreak \noindent 
    \begin{itemize}
        \item Good attendance
        \item Promptness
        \item Proper attire
        \item Clean and safe work environment
        \item Appropriate voice
        \item Pride
    \end{itemize}


    \pagebreak
    \begin{large}
       \noindent \textbf{ } 
    \end{large}
    

    
    

\end{document}
