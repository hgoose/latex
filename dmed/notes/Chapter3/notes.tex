\documentclass{report}

\input{~/dev/latex/template/preamble.tex}
\input{~/dev/latex/template/macros.tex}

\title{\Huge{Chapter 3 Notes: Image Files}}
\author{\huge{Nathan Warner}}
\date{\huge{Jan 27, 2023}}

\begin{document}
    \maketitle

    \begin{Large}
        \noindent \textbf{Learning Outcomes:}
    \end{Large}
    \begin{itemize}
        \item Convey the difference between painting and drawing programs. 
        \item Demonstrate an understanding of file Extensions and file types.
        \item Recognize the role that compression and Resolution play in file sizes.
    \end{itemize}
   
    \bigbreak \noindent \bigbreak \noindent 
    \begin{Large}
       \noindent \textbf{Define Key Terms:} 
    \end{Large}
    \begin{itemize}
        \item \textbf{Paint Program:} Genral term for graphics software that uses pixels to create an image.
            \item \textbf{Drawing Program:} Genaral term for graphics software that uses mathematically defined lines to create an image.
            \item \textbf{Pixels:} Specific color at a specific location in a matrix or grid.
            \item \textbf{Gui:} Graphical user interface, makes it possible to use a device such as a mouse to interact with the computer.
            \item \textbf{Bandwidth:} The speed at which a computer can transfer information along a network.
            \item \textbf{Compression:} The process of reducing the size of an image
            \item \textbf{Pixel Dimension:} The number of pixels in a row and column of a raster grid.
            \item \textbf{Resolution:} The density of pixels in an image.
            \item \textbf{Resampling:} Adding or removing pixels during the process of resizing.
            \item \textbf{Aspect Ratio:} The ratio of the width to the height of an image.
    \end{itemize}

    \bigbreak \noindent \bigbreak \noindent 
    \begin{Large}
        \textbf{Key Concepts:}
    \end{Large} 
    \begin{enumerate}
        \item  Image programs can be categorized into two groups: paint programs and drawing programs. 
        \item Paint programs produce images using pixels. Each pixel consists of a spe-cific color. Images produced by paint programs are called raster images or bitmapped images. 
        \item Drawing programs use vectors or lines to produce an image. The vectors are created using a series of mathematical points. Images produced by drawing programs are called vector images. 
        \item Extensions for the most common raster file types are .bmp, .tif, .gif, .jpg,and .png. 
        \item Extensions for the file formats that are typically used for the Web are .gif, .jpg, and .png. 
        \item Extensions for the most common vector file formats are .svg and .eps.  
        \item Different file formats are appropriate for different situations. 
        \item JPG image file sizes are reduced using a lossy compression algorithm  that removes unnecessary pixels. GIF, PNG, and TIF use a lossless  compression. 
        \item Resolution and pixel density help determine the size of a file. 
        \item Resizing is best done maintaining the aspect ratio.
    \end{enumerate}

    \pagebreak
    \begin{Large}
        \noindent \textbf{Distinguish between Graphic Programs:}
    \end{Large}
    
    \bigbreak \noindent 
    There are two types of Graphic Programs: \textbf{drawing} and \textbf{painting}. \textbf{Paint Programs}
    create images by using pixels. \textbf{drawing programs} use vectors or lines to produce an image.
    These lines are created using a series of mathematical points that can be changes without
    distorting the image.

    \bigbreak \noindent \bigbreak \noindent \bigbreak \noindent 
    \begin{large}
        \textbf{Raster-Based Paint Programs:}
    \end{large}
    \bigbreak \noindent 
    Paint Programs create images by assigning each pixel to a point on a grid of x and y coordinates. 
    This grid is called a \textbf{raster.} Images that are created using this grid of pixels 
    are called \textbf{raster images} or \textbf{bitmaps}.

    \bigbreak \noindent  
    Photographs use raster graphics because of the wide range of colors that are available for the pixels. 
    However because of the nature of pixels, enlarging these raster images can cause pixelation within the image.
    Making raster images smaller can cause a loss of sharpness.
    
    \bigbreak \noindent 
    \nt{Adobe Photoshop is a popular software that is used to manage raster images. }

    \bigbreak \noindent  \bigbreak \noindent \bigbreak \noindent 
    \begin{Large}
        \textbf{Determinig Image file formats:}
    \end{Large}

    \bigbreak \noindent 
    A file extension is the 3 or 4 letters that follow the dot at the end of the file name.
    This extension indicates the file format, and which programs can open the file.
    
    \bigbreak \noindent \bigbreak \noindent 
    \begin{Large}
        \textbf{Types of file Extensions:}
    \end{Large}

    \bigbreak \noindent 
    \begin{itemize}
        \item \textbf{BMP:} Bitmaps, One of the earliest file types. these files 
            are usually placed in word processing documents. Bitmap file sizes are usually 
            quite large even though they are limited to 256 colors because they are created without
            any kind of compression.
        \item \textbf{JPG:} Joint Photographic Experts Group, These files use up to 16 million
            colors. JPG images reproduce the quality, color, and detail found in photographs or graphics
            that use blends and gradients. Most cameras save images as JPG to conserve memory space
            on the cameras storage. JPG in the most common nonnative raster file format in use today.
        \item \textbf{GIF:} Graphics Interchange Format, GIFs are compressed and only use 256 colors. 
            The file sizes for GIFs are usually quite small. Much of the quality is lost if 
            files are saved as GIFs. However, they are suitable for line drawings, images with
            transparent backgrounds, and animated figures. GIFs are commonly used in web page design.
        \item \textbf{TIF:} Tagged Image Format. These files work well in all enviornments. Like Bitmaps, 
            these files are quite large. TIF files can support up to 16 million colors and are 
            often used in print documents. Some cameras save photos as TIF aswell as the standard JPG
        \item \textbf{PNG:} Portable Network Graphics. PNGs are a popular choice for use on the internet.
            They retain 16 million colors and supports transparency. Often used to replace GIFs because 
            they are quite small and support a large ammount of colors.
        \item \textbf{EPS:} Encapsulated Postscript. This is a general purpose vector file format
            that has both the vector image data and a screen preview in the same file. 
            It is most commonly used for printing purposes.
        \item \textbf{SVG:} Scaleable Vector Graphics. This is a vector format designed for use 
            on the web. SVG images are created using HTML code. Software such as Adobe Illustrator
            can be used to create an image and covert it to SVG. These files are a popular choice for 
            mobile devices because of its small file size.
    \end{itemize}

    \pagebreak
    \begin{Large}
        \noindent \textbf{Managing Image Sizes:}       
    \end{Large} 

    \bigbreak \noindent 
    There are two ways to look at image sizes. The first is the ammount of space the 
    image takes up on disk. The other is the visual image size. Raster images that are 
    visually large take up more space on disk than smaller ones. Vector images
    do not change their storage requirements based upon size.

    \bigbreak \noindent 
    It is important to keep file size in mind. Some websites and email services restrict the size of image files that can be uploaded.
    At a minimum, huge attached images can slow email delivery. 

    \bigbreak \noindent 
    File size is a consideration when adding images to a website. The time it takes 
    to download a graphic and display it in a browser depends on both the size of the file and thetype of 
    internet connection. Dial up connections that require a modem and telephone
    recieve data over a narrow Bandwidth. Broadband connections using DSL, cable, or wireless
    connections are much faster because they have a larger Bandwidth.

    \bigbreak \noindent \bigbreak \noindent 
    \begin{Large}
        \textbf{Compression:}
    \end{Large}

    \bigbreak \noindent 
    To reduce image file sizes, several algorithms have been written to reduce, or compress 
    the size of an image file. 

    \bigbreak \noindent 
    There are \textbf{Two} types of compression
    \begin{itemize}
        \item Lossless
        \item Lossy
    \end{itemize}

    \bigbreak \noindent 
    Lossless compression reduces the file size without losing any pixels. Files saved as .png, .gif, and .tif
    use lossless compression
    \bigbreak \noindent 
    Lossy compression deletes or changes some pixels when saving. The .jpg file format uses 
    lossy compression. It is important to know that each time you change a lossy image 
    and then save it, the file is compressed again. If you save several times in this way, 
    you will see that image has begun to degrade. If you make all your changes in a native format, 
    and then save the last copy as a jpg, you will not encounter this problem. 

    \bigbreak \noindent \bigbreak \noindent \bigbreak \noindent  
    \begin{Large}
        \textbf{Resolution, Resizing, and Resampling}
    \end{Large}

    \bigbreak \noindent 
    Bitmaps have visual sizes measured in two ways:
    \begin{itemize}
        \item physical size.
        \item number of pixels in one inch. 
    \end{itemize}

    \bigbreak \noindent 
    Physical sizes are measured in pixel Dimension. 

    \bigbreak \noindent 
    The Resolution of an image is measured in pixel density, or pixels per inch (ppi). Images 
    with a higher ppi will produce a greater quality image. If you are sending an image using
    email, you will want to use a lower resolution. If you are printing an image, you will
    want the highest resolution possible.

    \bigbreak \noindent 
    \textbf{Resampling} is an option that allows you to add or remove pixels when you resize an 
    image. When resampling is on, changing the size of the image will not change its resolution, but 
    it will become larger or smaller. If you turn off resampling, changing the physical dimension
    of image will change its resolution.
\end{document}

